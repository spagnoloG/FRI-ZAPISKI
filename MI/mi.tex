\documentclass{article}
\usepackage[margin=0.15cm]{geometry}
\usepackage{amsmath}
\usepackage{multicol}
\usepackage{hyperref}
\usepackage{amssymb}


\begin{document}

\begin{center}
    {\small MI/FRI \par}
\end{center}

\begin{multicols}{2}

\section{\underline{Vektorji in matrike}}

\textbf{1.1} Vektor je \textit{urejena n-terica stevil}, ki jo obicajno
zapisemo kot stolpec\smallskip
\begin{center}
    $\vec{x}$ =
    $\begin{bmatrix}
        x_{1}\\
        \vdots \\
        x_{n}\\
    \end{bmatrix}$
\end{center}

\textbf{1.2} Produkt \textit{vektorja} $\vec{x}$ s skalarjem $\alpha$ je vektor
\begin{center}
    $\alpha \vec{x}$ =
    $\alpha$
    $\begin{bmatrix}
        x_{1}\\
        \vdots \\
        x_{n}\\
    \end{bmatrix}$ =
    $\begin{bmatrix}
        \alpha x_{1}\\
        \vdots \\
        \alpha x_{n}\\
    \end{bmatrix}$
\end{center}

\textbf{1.3} Vsota \textit{vektorjev} $\vec{x}$ in $\vec{y}$ je vektor
\begin{center}
    $\vec{x} + \vec{y} = 
    \begin{bmatrix}
        x_{1}\\
        \vdots \\
        x_{n}\\
    \end{bmatrix} +
    \begin{bmatrix}
        y_{1}\\
        \vdots \\
        y_{n}\\
    \end{bmatrix} =
    \begin{bmatrix}
        x_{1}  +  y_{1}\\
        \vdots\\
        x_{n} + y_{n}\\
    \end{bmatrix} 
    $
\end{center}

\textbf{1.4} Nicelni vektor $\vec{0}$ je tisti vektor, za katerega
je $\vec{a} + \vec{0} = \vec{0} + \vec{a} = \vec{a}$ za vsak vektor
$\vec{a}$. Vse komponente nicelnega vektorja so enake 0. Vsakemu vektorju
$\vec{a}$ priprada nasprotni vektor -$\vec{a}$, tako da je $\vec{a} + (-\vec{a}) = \vec{0}$
Razlika vektorjev $\vec{a}$ in $\vec{b}$ je vsota $\vec{a} + (-\vec{b})$ in jo
navadno zapisemo kot  $\vec{a} - \vec{b}$.

\textbf{Lastnosti vektorske vsote}
\begin{itemize}
    \item $\vec{a} + \vec{b} = \vec{b} + \vec{a}$ (komutativnost)
    \item $\vec{a} + (\vec{b} + \vec{c}) = (\vec{a} + \vec{b}) + \vec{c}$ (asociativnost)
    \item $a(\vec{a} + \vec{b}) = a\vec{a} + a\vec{b}$ (distributivnost)
\end{itemize}

\textbf{1.5} Linearna kombinacija vektorjev $\vec{x}$ in $\vec{y}$ je vsota
\begin{center}
    $a\vec{x} + b\vec{y}$
\end{center}

\textbf{1.6} Skalarni produkt vektorjev\\
\begin{center}
    $\begin{bmatrix} 
        x_{1}\\ 
        \vdots\\ 
        x_{n}\\
    \end{bmatrix}$ in
    $\begin{bmatrix} 
        y_{1}\\ 
        \vdots\\ 
        y_{n}\\
    \end{bmatrix}$ je stevilo    
\end{center}
\begin{center}
    $\vec{x} \cdot \vec{y} = x_{1}y_{1} + x_{2}y_{2} + \dots + x_{n}y_{n}$
\end{center} \textit{alternativno:}
\begin{center}
    $\vec{x} \cdot \vec{y} = ||\vec{x}|| ||\vec{y}|| \cos \phi$
\end{center}

\textbf{Lastnosti skalarnega produkta}
\begin{itemize}
    \item $\vec{x} \cdot \vec{y} = \vec{y} \cdot \vec{x}$ (komutativnost)
    \item $\vec{x} \cdot (\vec{y} + \vec{z}) = \vec{x} \cdot \vec{y} + \vec{x} \cdot \vec{z}$ (aditivnost)
    \item $\vec{x} \cdot (a \vec{y}) = a(\vec{x} \cdot \vec{y}) = (a \vec{x}) \cdot \vec{y}$ (homogenost)
    \item $\forall \vec{x}$ \textit{velja} $\vec{x} \cdot \vec{x} \geq 0$
\end{itemize}

\textbf{1.7} Dolzina vektorja $\vec{x}$ je
\begin{center}
    $||\vec{x}|| = \sqrt{\vec{x} \cdot \vec{x}}$
\end{center}

\textbf{1.8} Enotski vektor je vektor z dolzino 1.

\textbf{1.9} Za poljubna vektorja $\vec{u}, \vec{v} \in R^{n}$ velja:
\begin{center}
    $|\vec{u} \cdot \vec{v}| \leq ||\vec{u}||||\vec{v}||$,
\end{center}
enakost velja, v primeru, da sta vektorja vzporedna.


\textbf{1.10} Za poljubna vektorja $\vec{u}, \vec{v} \in R^{n}$ velja:
\begin{center}
    $||\vec{u} + \vec{v}|| \leq ||\vec{u}||+||\vec{v}||$.
\end{center}

\textbf{1.11} Vektorja $\vec{x}$ in $\vec{y}$ sta ortogonalna
(pravokotna) natakno takrat, kadar je
\begin{center}
    $\vec{x} \cdot \vec{y} = $ 0    
\end{center}

\textbf{1.12} Ce je $\phi$ kot med vektorjema $\vec{x}$ in $\vec{y}$, potem je
\begin{center}
    $\dfrac{\vec{x} \cdot \vec{y}}{||\vec{x}|| ||\vec{y}||} =
    \cos \phi$
\end{center}

\textbf{1.13} Vektorski produkt:
\begin{center}
    $\vec{a} \times \vec{b} = (a_{2}b_{3} - a_{3}b_{2}) \textbf{i}$ +
    $(a_{3}b_{1} - a_{1}b_{3}) \textbf{j} + (a_{1}b_{2} - a_{2}b_{1}) \textbf{k}$
\end{center}

\textbf{Lastnosti vektorskega produkta}
\begin{itemize}
    \item $\vec{a} \times (\vec{b} + \vec{c}) = \vec{a} \times \vec{b} + \vec{a} \times \vec{c}$ (aditivnost)
    \item $\vec{b} \times \vec{a} = -\vec{a} \times \vec{b}$ (!komutativnost)
    \item $ (a \vec{a}) \times \vec{b} = a(\vec{a} \times \vec{b}) =  \vec{a} \times (a \vec{b})$ (homogenost)
    \item $\vec{a} \times \vec{a} = 0$
    \item $\vec{a} \times \vec{b}$  \textit{je}  $\perp$ \textit{na vektorja} $\vec{a}$ \textit{in} $\vec{b}$
    \item $||\vec{a} \times \vec{b}|| = ||\vec{a}|| ||\vec{b}|| \sin \phi$
    \item Dolzina vektorskega produkta je ploscina paralelograma, katerega vektorja oklepata 
\end{itemize}

\textbf{1.14} Mesani produkt($\vec{a}, \vec{b}, \vec{c}$) vektorjev
$\vec{a}, \vec{b}$ in $\vec{c}$ v $R^{3}$ je skalarni produkt vektorjev
$\vec{a} \times \vec{b}$ in $\vec{c}$:
\begin{center}
    $(\vec{a}, \vec{b}, \vec{c}) = (\vec{a} \times \vec{b})\cdot \vec{c}$
\end{center}

\textbf{Lastnosti mesanega produkta}
\begin{itemize}
    \item $(\vec{a}, \vec{b}, \vec{c}) = (\vec{b}, \vec{c}, \vec{a}) = (\vec{c}, \vec{a}, \vec{b})$
    \item $(x\vec{a}, \vec{b}, \vec{c}) = x(\vec{a}, \vec{b}, \vec{c})$ (homogenost)
    \item $(\vec{a}, \vec{u} + \vec{v}, \vec{c}) = (\vec{a}, \vec{u}, \vec{c}) + (\vec{a}, \vec{v}, \vec{c})$
    \item Absolutna vrednost mesanega produkta ($\vec{a}, \vec{b}, \vec{c}$) je enaka prostornini paralepipeda
\end{itemize}
    
\textbf{Premice v $R^{3}$} \\
Premico določata smerni vektor $\vec{p} = [a, b, c]^{T}$ in točka $A(x_0, y_0, z_0)$.
\begin{itemize}
    \item Parametrična oblika:
        $\vec{r} = \vec{r}_{A} + t\vec{p}$, $t \in R$
    \item Kanonična oblika:
        $\dfrac{x - x_{0}}{a} = \dfrac{y - y_{0}}{b} = \dfrac{z - z_{0}}{c}$
\end{itemize}

\textbf{Ravnine v $R^{3}$} \\
Ravnina z normalo $\vec{n} = [a, b, c]^T$ skozi točko $A(x_0, y_0, z_0)$ ima enačbo
\begin{center}
    $(\vec{r} - \vec{r}_A) \cdot \vec{n} = 0$
\end{center}
oziroma
\begin{center}
    $ax + by + cz = d$
\end{center}

\textbf{Razdalje}\\
Razdalja od tocke $P$ do ravnine, v kateri lezi tocka $A$ :
\begin{center}
    $\cos\phi = \dfrac{\vec{n} \cdot ( \vec{r_{P}} - \vec{r_{A}})} {||\vec{n}|| ||\vec{r_{P}} - \vec{r_{A}}||}$ oz.
    $d = |\dfrac{\vec{n}}{||\vec{n}||} ( \vec{r_{P}} - \vec{r_{A}})|$
\end{center}
Razdalja od tocke $P$ do premice, katera gre skozi tocko $A$:
\begin{center}
    $d = \dfrac{||\vec{e} \times ( \vec{r_{P}} - \vec{r_{A}})||}{||\vec{e}||}$
\end{center}

\textbf{Projekcije vektorjev}\\
Naj bo $proj_{\vec{a}}\vec{b} = \vec{x}$ projekcija vektorja $\vec{b}$ na vektor $\vec{a}$.
Izracunamo jo po sledeci formuli:
\begin{center}
    \begin{math}
        proj_{\vec{a}}\vec{b} = \frac{\vec{a}\vec{b}}{\vec{a}\vec{a}} \vec{a}
    \end{math}
\end{center}

\textbf{1.15} Matrika dimenzije $m \times n$ je tabela $m \times n$ stevil, urejenih
v $m$ vrstic in $n$ stolpcev:
\begin{center}
    $A^{m \times n} =$
    $\begin{bmatrix}
        x_{11} & x_{12} & x_{13} & \dots  & x_{1n} \\
        x_{21} & x_{22} & x_{23} & \dots  & x_{2n} \\
        \vdots & \vdots & \vdots & \ddots & \vdots \\
        x_{m1} & x_{m2} & x_{m3} & \dots  & x_{mn}
    \end{bmatrix}$
\end{center}

\textbf{1.16} Matrika, katere elementi so enaki nic povsod
zunaj glavne diagonale, se imenuje diagonalna matrika. Za
diagonalno matriko je $a_{ij} = 0$, kadarkoli velja $i \neq j$

\textbf{1.17} Matrika $A^{n \times n}$ je spodnjetrikotna, kadar
so vsi elementi nad glavno diagonalo enaki 0:
\begin{center}
    $a_{ij} = 0$  \textit{kadar je} $i < j$
\end{center}

\textbf{1.18} Matrika $A^{n \times n}$ je zgornjetrikotna, kadar
so vsi elementi pod glavno diagonalo enaki 0:
\begin{center}
    $a_{ij} = 0$  \textit{kadar je} $i > j$
\end{center}

\textbf{1.19} Matrika je trikotna, ce je zgornjetrikotna ali spodnjetrikotna.

\textbf{1.20} Dve matriki $A$ in $B$ sta enaki natanko takrat,
kadar imata enaki dimenziji in kadar so na istih mestih v obeh
matrikah enaki elementi:
\begin{center}
    $A^{m \times n} = B^{p \times q} \implies m=p$ in $n=q$,\\
    $a_{ij} = b_{ij}$ \textit{za vsak} $i= 1,...,m$ in $j=1,...,n$ 
\end{center}

\textbf{1.21} Produkt matrike s skalarjem dobimo tako, da 
vsak element matrike pomnozimo s $skalarjem$
\begin{center}
    $aA^{m \times n} =$
    $\begin{bmatrix}
        ax_{11} & ax_{12} & ax_{13} & \dots  & ax_{1n} \\
        ax_{21} & ax_{22} & ax_{23} & \dots  & ax_{2n} \\
        \vdots  & \vdots  & \vdots  & \ddots  & \vdots \\
        ax_{m1} & ax_{m2} & ax_{m3} & \dots  & ax_{mn}
    \end{bmatrix}$
\end{center}

\textbf{1.22} Vsoto dveh matrik enake dimenzije dobimo tako,
da sestejemo istolezne elemente obeh matrik:
\begin{center}
    $A + B =$
    $\begin{bmatrix}
        a_{11} + b_{11} & ax_{12} + b_{12}  & \dots  & ax_{1n} + b_{1n} \\
        a_{21} + b_{21} & ax_{22} + b_{22}  & \dots  & ax_{2n} + b_{2n}\\
        \vdots          & \vdots            & \ddots & \vdots \\
        a_{m1} + b_{m1} & ax_{m2} + b_{m3} & \dots  & ax_{mn} + b_{mn}
    \end{bmatrix}$
\end{center}

\textbf{Osnovne matricne operacije}
\begin{itemize}
    \item $A + B = B + A$ (komutativnost)
    \item $(A + B) + C = A + (B + C)$ (asociativnost)
    \item $a(A + B) = aA + aB$ (mnozenje s skalarjem)
    \item $A + (-A) = 0$
    \item $x(yA) = (xy)A$ \textit{in} $1 \cdot A = A$
\end{itemize}

\textbf{1.23} Transponirana matrika k matriki A reda $m \times n$
je matrika reda $n \times m$
\begin{center}
    $A =$
    $\begin{bmatrix}
        x_{11} & x_{12} & \dots  & x_{1n} \\
        x_{21} & x_{22} & \dots  & x_{2n} \\
        \vdots & \vdots & \ddots & \vdots \\
        x_{m1} & x_{m2} & \dots  & x_{mn}
    \end{bmatrix}$\\
    \smallskip
    $A^{T} =$
    $\begin{bmatrix}
        x_{11} & x_{21} & \dots  & x_{m1} \\
        x_{12} & x_{22} & \dots  & x_{m2} \\
        \vdots & \vdots & \ddots & \vdots \\
        x_{1n} & x_{2n} & \dots  & x_{mn}
    \end{bmatrix}$
\end{center}

\textbf{Lastnosti transponiranja matrik}
\begin{itemize}
    \item $(A + B)^{T} = A^{T} + B^{T}$
    \item $(A \cdot B)^{T} = B^{T} \cdot A^{T}$
    \item $(xA)^{T} = xA^{T}$
    \item $(A^{T})^{T} = A$
\end{itemize}

\textbf{1.24} Produkt matrike A in vektorja $\vec{x}$ je
linearna kombinacija stolpcev matrike A, utezi linearne
kombinacije so komponente vektorja $\vec{x}$:
\begin{center}
    $A \vec{x} =
    \begin{bmatrix}
                &         & \\
        \vec{u} & \vec{v} & \vec{w} \\
                &         & \\
    \end{bmatrix}
    \cdot
    \begin{bmatrix}
        a\\
        b\\
        c
    \end{bmatrix} =$
    $a\vec{u} + b\vec{v} + c\vec{w}$
\end{center}

\textbf{1.25} Produkt vrstice $\vec{x}$ z matriko A je
linearna kombinacija vrstic matrike A, koeficienti linearne
kombinacije so komponente vrstice $\vec{y}$:
\begin{center}
    $\vec{y} \cdot A =
    \begin{bmatrix}
        y_{1}, y_{2}, y_{3}
    \end{bmatrix} \cdot
    \begin{bmatrix}
        \vec{u}\\
        \vec{v}\\
        \vec{w}
    \end{bmatrix} =
    \begin{bmatrix}
        y_{1}\vec{u}\\
        y_{2}\vec{v}\\
        y_{3}\vec{w}
    \end{bmatrix}
    $
\end{center}

\textbf{1.26} Produkt matrik A in B je matrika, katere stolpci
so zaporedoma produkti matrike A s stolpci matrike B:
\begin{center}
    $AB = A
    \begin{bmatrix}
        b_{1}, b_{2}, \dots ,b_{n}
    \end{bmatrix} =
    \begin{bmatrix}
        Ab_{1}, Ab_{2}, \dots ,Ab_{n}
    \end{bmatrix}
    $
\end{center}

\textbf{1.27} Element $c_{ij}$ v $i-ti$ vrstici in $j-tem$ stolpcu
produkta C = AB je skalarni produkt $i-te$ vrstice A in $j-tega$
stolpca matrike B
\begin{center}
    $c_{ij} =
    \sum_{k=1}^{n} a_{ik}b_{kj}
    $
\end{center}

\textbf{1.28} Produkt matrik A in B je matrika, katere vrstice
so zaporedoma produkti vrstic matrike A z matriko B:
\begin{center}
    $
    \begin{bmatrix}
        i-ta\; vrstica\; A
    \end{bmatrix}B =
    \begin{bmatrix}
        i-ta\; vrstica\; AB
    \end{bmatrix}
    $
\end{center}

\textbf{Lastnosti matricnega produkta}
\begin{itemize}
    \item $AB \neq BA$ (!komutativnost)
    \item $(xA)B = x(AB) = A(xB)$ (homogenost)
    \item $C(A + B) = CA + CB$ (distributivnost)
    \item $A(BC) = (AB)C$ (asociativnost)
    \item $(AB)^{T} = B^{T}A^{T}$
\end{itemize}
V splosnem; komutativnost matricnega mnozenja velja
samo, ko sta matriki diagonalizabilni.

\textbf{1.29} Vrstice matrike A z $n$ stolpci naj bodo
$a^{1}, \dots, a^{n}$, stolpci matrike B z $n$ vrsticami pa
$b_{1}, \dots, b_{n}$. Potem je
\begin{center}
    $AB = a^{1}b_{1} + \dots + a^{n}b_{n}$
\end{center}

\textbf{1.30} Ce delitev na bloke v matriki A ustreza delitvi v matirki B,
potem lahko matriki pomnozimo blocno:
\begin{center}
    $\begin{bmatrix}
        A_{11} & A_{12}\\
        A_{21} & A_{22}
    \end{bmatrix}
    \begin{bmatrix}
        B_{11} & B_{12}\\
        B_{21} & B_{22}
    \end{bmatrix} =
    \begin{bmatrix}
        A_{11}B_{11} + A_{12}B_{21} & A_{11}B_{12} + A_{12}B_{22}\\
        A_{21}B_{11} + A_{22}B_{21} & A_{21}B_{12} + A_{22}B_{22}
    \end{bmatrix}$
\end{center}

\textbf{1.31} Kvadratna matrika $I_{k}$ reda $k \times k$, ki ima vse diagonalne
elemente enake 1, vse ostale elemente pa 0 ima lastnost, da za vsako matriko A
reda $m \times n$ velja $AI_{n} = A$ in $I_{m}A = A$. Matrika $I_{k}$ se imenuje
enotska ali identicna matirka.
\begin{center}
    $I_{k}=
    \begin{bmatrix}
        1 & 0 & \hdots & 0\\
        0 & 1 & \hdots & 0 \\
        \vdots & \vdots & \ddots & \vdots\\
        0 & 0 & \hdots & 1
    \end{bmatrix} 
    $
\end{center}

\section{\underline{Sistemi linearnih enacb}}

\textbf{2.1} Kvadratna matrika A je obrnljiva, ce obstaja taka matrika
$A^{-1}$, da je
\begin{center}
    $AA^{-1} = I\;
    in\;
    A^{-1}A = I
    $
\end{center}
Matrika $A^{-1}$ (ce obstaja) se imenuje matriki A inverzna matrika.
Matrika, ki ni obrnljiva, je singularna. Matrika \textbf{NI} obrnljiva, kadar je
$rang(A) < n$ !

\textbf{2.2} Kvadratna matirka reda $n$ je obrnljiva natanko tedaj, ko pri
gaussovi eliminaciji dobimo $n$ pivotov.

\textbf{2.3} Vsaka obrnljiva matrika ima eno samo inverzno matriko.

\textbf{2.4} Inverzna matrika inverzne matrike $A^{-1}$ je matrika A
\begin{center}
    $(A^{-1})^{-1} = A$
\end{center}

\textbf{2.5} Ce je matrika A obrnljiva, potem ima sistem enacb
$A\vec{x} = \vec{b}$ edino resitev $\vec{x} = A^{-1} \vec{b}$

\textbf{2.6} Ce obstaja nenicelna resitev $\vec{x}$ enacbe $A\vec{x} = \vec{0}$,
matrika A ni obrnljiva(je singularna).

\textbf{2.7} Ce sta matirki A in B istega reda obrnljivi, je obrnljiv tudi
produkt $A \cdot B$ in
\begin{center}
    $(A \cdot B)^{-1} =
    B^{-1} \cdot A^{-1}
    $
\end{center}

\textbf{Pozor!} Pravilo
\begin{center}
    $(AB)^{p} = A^{p}B^{p}$
\end{center}
velja le v primeru, ko matriki A in B komutirata, torej $AB = BA$.

\textbf{2.8} Inverz transponirane matrike je transponirana matrika inverza
\begin{center}
    $(A^{T})^{-1} = (A^{-1})^{T}$
\end{center}

\textbf{2.9} Inverz diagonalne matrike z diagonalnimi elementi $a_{ii}$ je
diagonalna matrika, ki ima na diagonali elemente $a_{ii}^{-1}$
\begin{center}
    $\begin{bmatrix}
        a_{11} &        & 0\\
               & \ddots &\\
        0      &        & a_{nn}
    \end{bmatrix}=
    \begin{bmatrix}
        a_{11}^{-1} &        & 0\\
                    & \ddots &\\
        0           &        & a_{nn}^{-1}
    \end{bmatrix}
    $
\end{center}

\textbf{2.10} Za izracun inverza matrike A, uporabimo gausovo eliminacijo nad
matriko $\begin{bmatrix}A|I\end{bmatrix}$
\begin{center}
    $\begin{bmatrix}A|I\end{bmatrix} =
    \begin{bmatrix}I|A^{-1}\end{bmatrix}
    $
\end{center}

\textbf{2.11} Matrika A je simetricna $\Leftrightarrow A^{T} = A$. Za elemente
$a_{ij}$ simetricne matirke velja $a_{ij} = a_{ji}$. Za simetricno matriko vedno velja,
da je kvadratna $A \in R^{n \times n}$.

\textbf{2.12} Ce je matrika A simetricna in obrnljiva, je tudi $A^{-1}$ simetricna.

\textbf{2.13} Ce je R poljubna (lahko tudi pravokotna) matrika, sta $R^{T}R$ in
$RR^{T}$ simetricni matriki.

\section{\underline{Vektorski prostori}}

\textbf{3.1} Realni vektorski prostor V je mnozica "vektorjev" skupaj z pravili za
\begin{itemize}
    \item sestevanje vektorjev,
    \item mnozenje vektorja z realnim stevilom (skalarjem)
\end{itemize}
Ce sta $\vec{x}$ in $\vec{y}$ poljubna vektorja v V, morajo biti v V tudi
\begin{itemize}
    \item vsota $\vec{x} + \vec{y}$ in 
    \item produkti $\alpha\vec{x}$ za vse $\alpha \in R$
\end{itemize}
V vektorskem prostoru V morajo biti tudi VSE linearne kombinacije
$\alpha\vec{x} + \beta\vec{y}$

\textbf{Pravila za operacije v vektorskih prostorih}\\
Operaciji sestevanja vektorjev in mnozenja vektorja s skalarjem v vektorskem prostoru
morajo zadoscati naslednjim pravilom:
\begin{itemize}
    \item $\vec{x} + \vec{y} = \vec{y} + \vec{x}$ (komutativnost)
    \item $\vec{x} + (\vec{y} + \vec{z}) = (\vec{x} + \vec{y}) + \vec{z}$ (asociativnost)
    \item obstaja en sam nenicelni vektor $\vec{0}$, da velja $\vec{x} + \vec{0} = \vec{x}$
    \item za vsak $\vec{x}$ obstaja natanko en $-\vec{x}$, da je $\vec{x} + (-\vec{x}) = \vec{0}$
    \item $1 \cdot \vec{x} = \vec{x}$
    \item $(\alpha\beta)\vec{x} = \alpha(\beta\vec{x})$
    \item $\alpha(\vec{x} + \vec{y}) = \alpha\vec{x} + \alpha\vec{y}$ (distributivnost)
    \item $(\alpha + \beta)\vec{x} = \alpha\vec{x} + \beta\vec{x}$
\end{itemize}

\textbf{3.2} Podmnozica U vektorskega prostora V je \textit{vektorski podprostor}, ce je za
vsak par vektorjev $\vec{x}$ in $\vec{y}$ iz U in vsako realno stevilo $\alpha$ tudi
\begin{itemize}
    \item $\vec{x} + \vec{y} \in U$ in
    \item $\alpha\vec{x} \in U$.
\end{itemize}

\textbf{3.3} Mnozica vektorjev U je vektorski podprostor natanko tedaj, ko je vsaka linearna
kombinacija vektorjev iz U tudi v U.

\textbf{Lastnosti vektorskih podprostorov}
\begin{itemize}
    \item Vsak vektorski podprostor nujno vsebuje nicelni vektor $\vec{0}$
    \item Presek dveh podprostorov vektorskega podprostora je tudi podprostor
\end{itemize}

\textbf{3.4} Stolpicni prostor C(A) matrike $A \in R^{m \times n}$ je tisti podprostor
vektorskega prostora $R^{m}$, ki vsebuje natanko vse linearne kombinacije stolpcev matrike A.\\
Izracunamo ga tako, da matriko A transponiramo in izvedemo operacijo gaussove eliminacije nad $A^{T}$. Vrstice katere ostanejo po gaussivi eliminaciji
so linearno neodvisni vektorji, kateri tvorijo stoplicni prostor matrike A, $C(A)$.
\textit{neformalno: linearna ogrinjaca stolpcev matrike (npr. ce imas 5 stolpcev pa lahko 2 zapises kot linearno kombinacijo ostalih 3 bo imel column space 3 elemente)}

\textbf{3.5} Sistem linearnih enacb $A\vec{x} = \vec{b}$ je reslijv natanko tedaj, ko je vektor
$\vec{b} \in C(A)$

\textbf{3.6} Naj bo matrika $A \in R^{m \times n}$. Mnozica resitev homogenega sistema linearnih
enacb je podprostor v vektorskem prostoru $R^{n}$.

\textbf{3.7} Mnozica vseh resitev sistema linearnih enacb $A\vec{x} = \vec{0}$ se imenuje nicelni
prostor matirke A. Oznacujemo ga z N(A).\\
\textit{neformalno: mnozica vektorjev, ki se z neko matriko zmnozijo v nicelni vektor. Matriko A samo eliminiras po gaussu in nato dobljene resitve enacis z 0.}

\textbf{3.8} Ce je matrika A kvadratna in obrnljiva, potem N(A) vsebuje samo vektor $\vec{0}$

\textbf{3.9} Matrika ima \textit{stopnicasto} obliko, kadar se vsaka od njenih vrstic zacne z vsaj eno
niclo vec kot prejsnja vrstica.

\textbf{3.10} Prvi element, razlicen od nic v vsaki vrstici, je \textit{pivot}. Stevilo pivotov v matriki
se imenuje rang matrike. Rang matrike A zapisemo kot $rang(A)$.

\textbf{3.11} Rang matrike ni vecji od stevila vrstic in ni vecji od stevila stolpcev matrike.

\textbf{3.12}
\begin{center}
    \textit{Stevilo prostih neznank matrike = st. stolpcev - rang matrike}   
\end{center}

\textbf{3.13}
\begin{enumerate}
    \item Visoka in ozka matrika $(m > n)$ ima poln stolpicni rang, kadar je $rang(A) = n$
    \item Nizka in siroka matrika $(m < n)$ ima poln vrsticni rang, kadar je $rang(A) = m$
    \item Kvadratna matrika $(n = m)$ ima poln rang, kadar je $rang(A) = m = n$
\end{enumerate}

\textbf{3.14} Za vsako matriko A s polnim stolpicnim rangom $r = n \leq m$, velja:
\begin{enumerate}
    \item Vsi stolpci A so pivotni stolpci
    \item Sistem enacb $A\vec{x} = \vec{0}$ nima prostih neznank, zato tudi nima posebnih resitev
    \item Nicelni prostor $N(A)$ vsebuje le nicelni vektor $N(A) = \{\vec{0}\}$
    \item Kadar ima sistem enacb $A\vec{x} = \vec{b}$ resitev(kar ni vedno res!), je resitev ena sama
    \item Reducirana vrsticna oblika matrike (A) se da zapisati kot
\end{enumerate}
\begin{center}
    $R =
    \begin{bmatrix}
        I\\
        0
    \end{bmatrix}
    \begin{bmatrix}
        n \times n\; enotska\; matrika\\
        m - n\; vrstic\; samih\; nicel\;
    \end{bmatrix}
    $
\end{center}

\textbf{3.15} Za vsako matriko A s polnim vrsticnim rangom $r = m \leq n$ velja:
\begin{enumerate}
    \item Vse vrstice so pivotne, ni prostih vrstic in U (stopnicasta oblika) in R(reducirana stopnicasta oblika) nimata nicelnih vrstic
    \item Sistem enacb $A\vec{x} = \vec{b}$ je resljiv za vsak vektor $\vec{b}$
    \item Sistem $A\vec{x} = \vec{b}$ ima $n-r = n-m$ prostih neznank, zato tudi prav toliko posebnih resitev
    \item Stolpicni prostor $C(A)$ je ves prostor $R^{m}$
\end{enumerate}

\textbf{3.16} Za vsako kvadratno matriko A polnega ranga (rang(A) = m = n) velja:
\begin{enumerate}
    \item Reducirana vrsticna oblika matrike A je enotska matrika
    \item Sistem enacb $A\vec{x} = \vec{b}$ ima natancno eno resitev za vsak vektor desnih strani $\vec{b}$
    \item Matrika A je obrnljiva
    \item Nicelni prostor matrike A je samo nicelni vektor $N(A) = \{\vec{0}\}$
    \item Stolpicni prostor matrike A je cel prostor $C(A) = R^{m}$
\end{enumerate}

\textbf{3.17} Vektorji $\vec{x_{1}}, \dots,\vec{x_{n}}$ so linearno neodvisni, ce je
\begin{center}
    $ 0\vec{x_{1}} + 0\vec{x_{2}} + \dots + 0\vec{x_{n}}$
\end{center}
edina njihova linearna kombinacija, ki je enaka vektorju $\vec{0}$. Vektorji $\vec{x_{1}}, \dots,\vec{x_{n}}$ so
linearno odvisni, \textit{ce niso linearno neodvisni}.

\textbf{3.18} Ce so vektorji \textit{odvisni}, lahko vsaj enega izrazimo z ostalimi.

\textbf{3.19} Ce je med vektorji  $\vec{u_{1}}, \dots,\vec{u_{n}}$ tudi nicelni vektor, so 
vektorji \textit{linearno odvisni}.

\textbf{3.20} Vsaka mnozica n vektorjev iz $R^{n}$ je odvisna, kadar je $n > m $.

\textbf{3.21} Stolpci matrike A so linearno neodvisni natanko tedaj, ko ima homogena enacba
$A\vec{x} = \vec{0}$ edino resitev $\vec{x} = \vec{0}$.

\textbf{3.22} Kadar je $rang(A) = n$, so stolpci matrike $A \in R^{m \times n}$ linearno
neodvisni. Kadar je pa $rang(A) < n$, so stolpci matrike $A \in R^{m \times n}$ linearno odvisni.

\textbf{3.23} Kadar je $rang(A) = m$, so vrstice matrike $A \in R^{m \times n}$ linearno neodvisne.
Kadar je pa $rang(A) < m$, so vrstice matrike $A \in R^{m \times n}$ linearno odvisne.

\textbf{3.24} Vrsticni prostor matrike A je podprostor v $R^{n}$, ki ga razpenjajo vrstice matrike A.

\textbf{3.25} Vrsticni prostor matrike A je $C(A^{T})$, stolpicni prostor matrike $A^{T}$.

\textbf{3.26} \textit{Baza vektorskega prostora} je mnozica vektorjev, ki
\begin{enumerate}
    \item je linearno neodvisna in
    \item napenja cel prostor.
\end{enumerate}

\textbf{3.27} Vsak vektor iz vektorskega prostora lahko na en sam nacin izrazimo
kot linearno kombinacijo baznih vektorjev.
 
\textbf{3.28} Vektorji $\vec{x_{1}}, \dots,\vec{x_{n}}$ so baza prostora $R^{n}$ natanko tedaj, kadar 
je matrika, sestavljena iz stolpcev $\vec{x_{1}}, \dots,\vec{x_{n}}$, obrnljiva.

\textbf{3.29} Prostor $R^{n}$ ima za $n > 0$ neskoncno mnogo razlicnih baz.

\textbf{3.30} Ce sta mnozici vekotrjev {$\vec{v_{1}}, \dots,\vec{v_{m}}$} in $\vec{u_{1}}, \dots,\vec{u_{n}}$
obe bazi istega vektorskega prostora, potem je $m = n \implies$ vse baze istega vektorskega prostora imajo
isto stevilo vektorjev.

\textbf{3.31} \textit{Dimenzija} vektroskega prostora je stevilo baznih vektorjev.

\textbf{3.32} Dimenziji stolpicnega prostora $C(A)$ in vrsticnega prostora $C(A^{T})$ sta enaki rangu matrike $A$
\begin{center}
    $dim(C(A)) = dim(C(A^{T})) = rang(A)$.
\end{center}

\textbf{3.33} Dimenzija nicelnega prostora $N(A)$ matrike A z $n$ stolpci in ranga $r$
je enaka $dim(N(A)) = n - r$.

\textbf{3.34} Stolpicni prostor $C(A)$ in vrsticni prostor $C(A^{T})$ imata oba dimenzijo r. Dimenzija
nicelnega prostora $N(A)$ je $n -r$, Dimenzija levega nicelnega prostora $N(A^{T})$ pa je $m - r$.

\textbf{3.35} Vsako matriko ranga 1 lahko zapisemo kot produkt(stolpcnega) vektorja z vrsticnim
vektorjem $A = \vec{u}\vec{v}^{T}$.

\section{\underline{Linearne preslikave}}

\textbf{4.1} Preslikava $A: U \rightarrow V$ je linearna, ce velja
\begin{enumerate}
    \item aditivnost: $A(\vec{u}_{1} + \vec{u}_{2}) = A\vec{u}_{1} + A\vec{u}_{2}$ za vse $\vec{u}_{1}, \vec{u}_{2} \in U$,
    \item homogenost: $A(\alpha \vec{u}) = \alpha(A\vec{u})$ za vse $\alpha \in R$ in $\vec{u} \in U$.
\end{enumerate}
Oziroma v enem koraku:
\begin{center}
    \begin{math}
        A(\alpha\vec{u}_{1} + \beta\vec{u}_{2}) = \alpha A(\vec{u}_{1}) + \beta A(\vec{u}_{2}).
    \end{math}
\end{center}
\textbf{Pozor!} Preslikava ni linearna, ce $A(\vec{0}) \neq  \vec{0}$.

\textbf{4.2} Preslikava $A: U \rightarrow V$ je linearna natanko tedaj, ko velja
\begin{center}
    $A(\alpha_{1}\vec{u}_{1} + \alpha_{2}\vec{u}_{2}) = \alpha_{1}A\vec{u}_{1} + \alpha_{2}A\vec{u}_{2}$
\end{center}
za vse $\alpha_{1}, \alpha_{2} \in R$ in vse $\vec{u}_{1}, \vec{u}_{2} \in U$.

\textbf{4.3} Ce je A \textit{linearna preslikava}, je $A\vec{0} = \vec{0}$.

\textbf{4.4} Naj bo $A: U \rightarrow V$ linearna preslikava in $\sum_{i=1}^{k} \alpha_{i}\vec{u}_{i}$
linearna kombinacija vektorjev. Potem je A($\sum_{i=1}^{k} \alpha_{i}\vec{u}_{i}$) = $\sum_{i=1}^{k} \alpha_{i}A\vec{u}_{i}$.

\textbf{4.5} Naj bo $\beta =$ $\{ \vec{u_{1}}, \dots,\vec{u_{n}}\}$ baza za vektorski prostor U. Potem je linearna
preslikava $A: U \rightarrow V$ natanko dolocena, ce poznamo slike baznih vektorjev.

\textbf{4.6} Naj bo $\beta =$ $\{\vec{u_{1}}, \dots,\vec{u_{n}}\}$ baza za U in $\{\vec{v_{1}}, \dots,\vec{v_{n}}\}$.
Potem obstaja natanko ena linearna preslikava $A: U \rightarrow V$, za katero je $A\vec{u}_{i} = \vec{v}_{i}$ za $i = 1, 2, \dots, n$.

\textbf{4.7} Naj bo $A: U \rightarrow V$ linearna preslikava. Potem mnozico
\begin{center}
    $ker A = \{ \vec{u} \in U; A\vec{u} = \vec{0}\}$
\end{center}
imenujemo \textit{jedro} linearne preslikave. Ker je $A\vec{0} = \vec{0}$, je $\vec{0} \in$ ker A za vse A.
Zato je jedro vedno neprazna mnozica.
\textit{Ce je matrika A$\phi$ \textbf{enotska} preslikava za } $\phi$, \textit{potem velja}
\begin{center}
    \begin{math}
        ker \phi = N(A).
    \end{math}
\end{center}

\textbf{4.8} Jedro linearne preslikave $A: U \rightarrow V$ je vektorski podprostor v U.

\textbf{4.9} Mnozico
\begin{center}
    $im\; A = \{ \vec{v} \in V; obstaja\; tak\; \vec{u} \in U,\; da\; je\; \vec{v} = A\vec{u} \}$
\end{center}
imenujemo \textit{slika} linearne preslikave $A: U \rightarrow V$.
\textit{Ce je matrika A$\phi$ \textbf{enotska} preslikava za } $\phi$, \textit{potem velja}
\begin{center}
    \begin{math}
        im \phi = C(A).
    \end{math}
\end{center}

\textbf{4.10} Ce je $A: U \rightarrow V$ linearna preslikava, potem je njena slika $im\; A$
vektorski podprostor v V.

\textbf{4.11} Ce je $A: U \rightarrow V$ linearna preslikava, in je rang matrike te preslikave v standardni bazi poln,
potem lahko sklepamo, da ima  ta preslikava \textbf{trivialno jedro}.


\section{\underline{Ortogonalnost}}

\textbf{5.1} Podprostora $U$ in $V$ vektorskega prostora sta med seboj ortogonalna,
ce je vsak vektor $\vec{u} \in U$ ortogonalen na vsak vektor $\vec{v} \in V$.

\textbf{5.2} Za vsako matriko $A \in R^{m \times n}$ velja:
\begin{enumerate}
    \item Nicelni prostor $N(A)$ in vrsticni prostor $C(A^{T})$ sta ortogonalna podprostora $R^{n}$
    \item Levi nicelni prostor $N(A^{T})$ in stolpicni prostor $C(A)$ sta ortogonalna podprostora prostora $R^{m}$.
\end{enumerate}

\textbf{5.3} Ortogonalni komplement $V^{\perp}$ podprostora V vsebuje VSE vektorje, ki so ortogonalni na V.

\textbf{5.4} Naj bo A matrika dimenzije $m \times n$.
\begin{itemize}
    \item Nicelni prostor $N(A)$ je ortogonalni\\ komplement vrsticnega prostora $C(A^{T})$ v prostoru $R^{n}$
    \item Levi nicelni prostor $N(A^{T})$ je ortogonalni komplement stolpicnega prostora $C(A)$ v prostoru $R^{m}$.
\end{itemize}
\textbf{krajse:}
\begin{center}
    $N(A)$ = $C(A^{T})^{\perp}$\\
    $N(A^{T})$ = $C(A)^{\perp}$ \\
    tukaj lahko vedno pomnozimo s komplementom, da dobimo npr.\\
    $N(A)^{\perp}$ = $C(A^{T})$
\end{center}
\textit{dodatek:}
\begin{center}
    $dim N(A) = st. stolpcev - rang(A)$\\
    $dim N(A^{T}) = st. vrstic - rang(A)$\\
    $dim C(A) = dim C(A^{T}) = rang(A)$
\end{center}

\textbf{5.5} Za vsak vektor $\vec{y}$ v stolpicnem prostoru $C(A)$ obstaja v vrsticnem prostoru $C(A^{T})$ en sam
vektor $\vec{x}$, da je $A\vec{x} = \vec{y}$.

\textbf{5.6} Ce so stolpci matrike A linearno neodvisni, je matrika $A^{T}A$ obrnljiva.

\textbf{5.7} Matrika P je projekcijska, kadar
\begin{itemize}
    \item je simetricna: $P^{T} = P$ in
    \item velja $P^{2} = P$.
\end{itemize}

\textbf{5.8} Ce je P projekcijska matrika, ki projecira na podprostor U, potem je $I -P$ projekcijska
matrika, ki projecira na $U^{\perp}$, ortogonalni komplement podprostora U.

\textbf{5.9} Vektorji $\vec{q_{1}}, \vec{q_{2}}, \dots, \vec{q_{n}}$ so ortonormiranim kadar so ortogonalni in imanjo vsi
dolzino 1, torej
\begin{center}
    $\vec{q_{i}}^{T}\vec{q_{i}} = $ \Bigg\{ 
    $\begin{matrix}
        0\;  ko\; je\; i \neq j\; pravokotni\; vektorji\\
        1\;  ko\; je\; i = j\; enotski\; vektorji
    \end{matrix}$
\end{center}
za matriko $Q =$ [$\vec{q_{1}}, \vec{q_{2}} \dots \vec{q_{n}}$]  velja $Q^{T}Q = I$.

\textbf{5.10} Vektorji $\vec{q_{1}}, \dots, \vec{q_{n}}$ naj bodo ortonormirani v prostoru $R^{m}$. Potem
za matriko
\begin{center}
    $Q = \begin{bmatrix}
        \vec{q_{1}} \vec{q_{2}} \dots \vec{q_{n}}
    \end{bmatrix}$
\end{center}
velja, da je $Q^{T}Q = I_{n}$ enotska matrika reda n.

\textbf{5.11} Matrika Q je ortogonalna, kadar je
\begin{enumerate}
    \item kvadratna in
    \item ima ortonormirane stolpce.
\end{enumerate}

\textbf{5.12} Ce je Q ortogonalna matirka, potem je obrnljiva in
\begin{center}
    $Q^{-1} = Q^{T}$\\
    $dimU^{\perp} = n - dimU$\\
    $(U^{\perp})^{\perp} = U$
\end{center}

\textbf{5.13} Mnozenje z ortogonalno matriko ohranja dolzino vektorjev in kote med njimi. Ce je Q
ortogonalna matrika, potem je 
\begin{center}
    $|| Q \vec{x} || = || \vec{x} ||$ za vsak vektor $\vec{x}$ in\\
    $(Q\vec{x})^{T}Q\vec{y} = \vec{x^{T}} \vec{y}$ za vsak vektor $\vec{x}$ in $\vec{y}$
\end{center}

\textbf{5.14} Ce sta $Q_{1}$ in $Q_{2}$ ortogonalni matriki, je tudi produkt $Q = Q_{1}Q_{2}$ ortogonalna
matrika.

\textbf{5.15 Gram-Schmidtova} ortogonalizacija. Za vhod uporabimo Linearno ogrinjaco linearno neodvisnih vekotrjev. Po 
gram-schmidtovi ortogonalizaciji pa dobimo paroma ortogonalne vektorje.
Postopek:
\begin{center}
    \begin{math}
        \vec{u}_{1} = \vec{v}_{1}
    \end{math}\\
    \begin{math}
        \vec{u}_{2} = \vec{v}_{2} - proj_{\vec{u}_{1}}\vec{v}_{2}
    \end{math}\\
    \begin{math}
        \vec{u}_{3} = \vec{v}_{3} - proj_{\vec{u}_{1}}\vec{v}_{3} - proj_{\vec{u}_{2}}\vec{v}_{3}
    \end{math}\\
    \begin{math}
        \vdots
    \end{math}
\end{center}
Po tem postopku dobimo paroma ortogonalne vektorje po Gram-Schmidtovi ortogonalizaciji.


\textbf{5.16 QR Razcep:} Iz linearno neodvisnih vektorjev $a_{1}, \dots, a_{n}$ z \textit{Gram-Schmidtovo} ortogonalizacijo
dobimo ortonormirane vektorje $q_{1}, \dots, q_{n}$. Matriki A in Q s temi stolpci zadoscajo enacbi $A = QR$, kjer
je R zgornjetrikotna matrika.
\begin{itemize}
    \item Najprej z Gram-Schmidtovo ortogonalizacijo poiscemo linearno neodvisne vektorje matrike A
    \item Vektorje normiramo in jih zapisemo v matriko Q.
    \item Matriko R dobimo tako, da matriko $Q^{T}$ pomnozimo z matriko $A$
    \begin{center}
        \begin{math}
            R = Q^{T}A
        \end{math}
    \end{center}
    Tako smo prisli do vseh elementov v QR razcepu matrike A.
\end{itemize}
Sedaj ko imamo izracunane vse elemente lahko zapisemo se projekcijsko matriko. To je matrika pravokotne projekcije na $C(Q) = C(A)$.
Njen izracun je preprost:
\begin{center}
    \begin{math}
        QQ^{T} = pravokotna\; projekcija\; na\; C(Q)\; = C(A)
    \end{math}
\end{center}
Sedaj lahko to projekcijsko matriko pomnozimo z desne s poljubnim vektorjem in ugotovimo kam se preslika v prostoru $C(A)$.
V nasprotnem primeru, ce bi pa zeleli imeti projekcijsko matriko, s katero bi radi videli kam se vektor preslika v prostoru $N(A^{T})$, bi pa od identicne matrike
odsteli projekcijsko matriko za $C(Q)$.
\begin{center}
    \begin{math}
        I - QQ^{T} = pravokotna\; projekcija\; na\; C(A)^{\perp}\; = N(A^{T})
    \end{math}
\end{center}

\textbf{5.17} Vektorski prostor $\iota$ je mnozica vseh neskoncnih zaporedij $\vec{u}$ s koncno
dolzino
\begin{center}
    $||\vec{u}||^{2} = \vec{u} \cdot \vec{u} = \vec{u_{1}}^{2} + \vec{u_{2}}^{2} + \dots < \infty$
\end{center}

\textbf{5.18 Predoloceni sistemi}
\begin{center}
    \begin{math}
        A^{T}A
        \begin{bmatrix}
            a\\
            b
        \end{bmatrix}
        = A^{T}\vec{f}
    \end{math}
\end{center}
Kjer je A matrika sistemov linearnih enacb in $\vec{f}$ vektor pricakovanih resitev
po gaussovi eliminaciji zgornje enacbe, dobimo spremenljivke, ki predstavljao najboljso aproksimacijo vseh kombinaicij rezultatov in vhodnih spremenljivk.

\section{\underline{Determinante}}

\textbf{6.1} Determinanta enotske matirke je\\ $det(I) = 1$.
\begin{center}
    \begin{math}
        \begin{vmatrix}
            1 & 0\\
            0 & 1
        \end{vmatrix}
        = 1\; in\;
        \begin{vmatrix}
            1 & & 0\\
            & \ddots &\\
            0 & & 1\\
        \end{vmatrix}
        = 1.
    \end{math}
\end{center}

\textbf{6.2} Determinanta spremeni predznak, ce med seboj zamenjamo dve vrstici.

Dodatna lastnost:

\[
    \left| \begin{array}{cc}
A & C \\
0 & B \\
\end{array} \right| = \det(A) \det(B)
\]

\textbf{6.3} Determinanta je linearna funkcija vsake vrstice posebej. To pomeni, da se
\begin{enumerate}
    \item determinanta pomnozi s faktorjem t, ce eno vrstico determinante(vsak element v tej vrstici)
    pomnozimo s faktorjem t.
    \begin{center}
        \begin{math}
            \begin{vmatrix}
                ta & tb\\
                c  & d\\
            \end{vmatrix}
            = t
            \begin{vmatrix}
                a & b\\
                c & d\\
            \end{vmatrix}
        \end{math}
    \end{center}
    \item determinanta je vsota dveh determinant, ki se razlikujeta le v eni vrstici,
    ce je v provitni determinanti ta vrstica vsota obeh vrstic, ostale vrstice pa so enake
    v vseh treh determinantah.
    \begin{center}
        \begin{math}
            \begin{vmatrix}
                a + a' & b + b'\\
                c      &      d\\
            \end{vmatrix} =
            \begin{vmatrix}
                a & b\\
                c & d\\
            \end{vmatrix} +
            \begin{vmatrix}
                a' & b'\\
                c & d\\
            \end{vmatrix}
        \end{math}
    \end{center}
\end{enumerate}

\textbf{Pozor!} Kadar mnozimo matriko A s skalarjem t, se vsak element matrike pomnozi s skalarjem.
Ko racunamo determinanto produkta matirke s skalarjem $tA$, skalar $t$ izpostavimo iz vsake vrstice posebej,
zato je $det(tA) = t^{n}det(A)$, kjer je $n$ stevilo vrstic (ali stolpcev) determinante.

\textbf{6.4} Matrika, ki ima dve enaki vrstici, ima determinanto enako 0.

\textbf{6.5} Ce v matriki od poljubne vrstice odstejemo mnogokratnik neke druge vrstice,
se njena determinanta ne spremeni.

\textbf{6.6} Naj bo $A$ poljubna kvadratna matirka $n \times n$ in $U$ njena vrsticno-stopnicasta
oblika, ki jo dobimo z \textit{Gaussovo eliminacijo}. Potem je 
\begin{center}
    $det(A) = \pm det(U)$.
\end{center}

\textbf{6.7} Determinanta, ki ima vrstico samih nicel, je enaka 0.

\textbf{6.8} Determinanta trikotne matrike $A$ je produkt diagonalnih elementov:
\begin{center}
    $det(A) = a_{11}a_{22} \hdots a_{nn}$.
\end{center}

\textbf{6.9} Determinanta singularne matrike je enaka 0, determinanta obrnljive matrike je razlicna od 0.

\textbf{6.10} Determinanta produkta dveh matrik je enaka produktu determinant obeh matrik:
\begin{center}
    $det(AB) = det(A)det(B)$.
\end{center}

\textbf{6.11} Determinanta inverzne matrike je enaka
\begin{center}
    $det(A^{-1}) = 1/det(A)$
\end{center}
in determinanta potence $A^{n}$ matrike A je
\begin{center}
    $det(A^{n}) = (det(A))^{n}$
\end{center}
ter determinanta transponirane matrike je enaka determinanti originalne matrike,
saj ko naredimo razvoj po vrsticah, pridemo do enakih elementov po diagonali.
\begin{center}
    $det(A) = det(A^{T})$.
\end{center}

\textbf{6.12} Transponirana matrika $A^{T}$ ima isto determinanto kot A.

\textbf{6.13 Recap dovoljenih operacij nad determinanto}
\begin{enumerate}
    \item Ce zamenjamo dve vrstici, se \textbf{spremeni} predznak determinante
    \item Vrednost determinante se ne spremeni, ce neki vrstici pristejemo poljuben veckratnik katerekoli druge vrstice.
    \item Ce vse elemente neke vrstice pomnozimo z istim stevilom $\alpha$, se vrednost determinante pomnozi z $\alpha$.
\end{enumerate}

\textbf{6.14} Vsaka lastnost, ki velja za vrstice determinante, velja tudi
za njene \textbf{stolpce}. Med drugim:
\begin{itemize}
    \item Determinanta spremeni predznak, ce med seboj zamenjamo dva stolpca
    \item Determinanta je enaka 0, ce sta dva stolpca enaka
    \item Determinanta je enaka 0, ce so v vsaj enem stolpcu same nicle.
\end{itemize}

\textbf{6.15 (kofaktorska formula)} Ce je A kvadratna matrika reda n,
njeno determinanto lahko izracunamo z razvojem po $i-ti$ vrstici
\begin{center}
    $det(A) = a_{i1}C_{i1} + a_{i2}C_{i2} + \hdots + a_{in}C_{in}$.
\end{center}
Kofaktorje $C_{ij}$ izracunamo kot $C_{ij} = (-1)^{i+j}D_{ij}$, kjer je $D_{ij}$ determinanta,
ki jo dobimo, ce v A izbrisemo i-to vrstico in j-ti stolpec.

\textbf{6.16} Inverzna matrika $A^{-1}$ matrike A je transponirana matrika kofaktorjev,
deljena z determinanto $|A|$:
\begin{center}
    $A^{-1} = \frac{C^{T}}{det(A)}$,
\end{center}
kjer je C matrika kofaktorjev matrike A.

\textbf{6.17} Ploscina paralelograma, dolocenega z vektorjema $\vec{a}$ in $\vec{b} \in R^{2}$ je
enaka det([$\vec{a} \vec{b}$]), to je absolutni vrednosti determinante s stolpcema $\vec{a}$ in $\vec{b}$.

\textbf{6.18} Mesani produkt vektorjev $\vec{a}$ in $\vec{b}$ in $\vec{c}$ je enak determinanti matrike, ki 
ima te tri vektorje kot stolpce.

\textbf{6.19} Naj bo A matrika $R^{n\times n}$
\begin{center}
    \begin{math}
        A\; je\; obrnljiva\; \iff detA \neq 0
    \end{math}
\end{center}
\begin{center}
    \begin{math}
        A^{-1}\; ne\; obstaja\; \iff detA = 0
    \end{math}
\end{center}

\section{\underline{L. vrednosti in vektorji}}

\textbf{7.1} Vektor $\vec{x} \neq \vec{0}$, za katerega je $A\vec{x} = \lambda \vec{x}$ lastni vektor. Stevilo
$\lambda$ je lastna vrednost.
\textbf{Pozor!} Nicelni vektor $\vec{0}$ ne more biti lastni vektor. Lahko pa je lastna vrednost enaka 0.

\textbf{7.2} Ce ima matrika A lastno vrednost $\lambda$ in lastni vektor $\vec{x}$, potem ima matrika
$A^{2}$ lastno vrednost $\lambda^{2}$ in isti lastni vektor $\vec{x}$.

\textbf{7.3} Ce ima matrika A lastno vrednost $\lambda$ in lastni vektor $\vec{x}$, potem ima
matrika $A^{k}$ lastno vrednost $\lambda^{k}$ in isti lastni vektor $\vec{x}$.

\textbf{7.4} Ce ima matrika A lastno vrednost $\lambda$ in lastni vektor $\vec{x}$, potem ima
inverzna matrika lastno vrednost $1 / \lambda$ in isti lastni vektor $\vec{x}$.

\textbf{7.5} Sled kvadratne matrike A reda $n$ je vsota njenih diagonalnih elementov.
\begin{center}
    \begin{math}
        sled(A) =
        \sum_{i=1}^{n} a_{ii} =
        a_{11} + \dots + a_{nn}
    \end{math}.
\end{center}

\textbf{7.6} Sled matrike je enaka vsoti vseh lastnih vrednosti, stetih z njihovo veckratnostjo.
Ce so $\lambda_{1}, \dots, \lambda_{n}$ lastne vrednosti matrike reda n, potem je sled enaka \textit{vsoti}
\begin{center}
    \begin{math}
        sled(A) =
        \sum_{i=1}^{n} \lambda_{i} =
        \lambda_{1} + \dots + \lambda_{n}
    \end{math},
\end{center}
determinanta matrike pa \textit{produktu} lastnih vrednosti
\begin{center}
    \begin{math}
        det(A) =
        \prod_{i=1}^{n} \lambda_{i} =
        \lambda_{1} \dots  \lambda_{n}
    \end{math}.
\end{center}

\textbf{Lastnosti sledi} Za matrike \( A, B, P \in \mathbb{R}^{n \times n} \) velja
\begin{enumerate}
    \item \( \text{tr}(\alpha A) = \alpha \text{tr}(A) \),
    \item \( \text{tr}(A + B) = \text{tr}(A) + \text{tr}(B) \),
    \item \( \text{tr}(A^T) = \text{tr}(A) \),
    \item \( \text{tr}(AB) = \text{tr}(BA) \),
    \item \( \text{tr}(PAP^{-1}) = \text{tr}(A) \) za vsako obrnljivo matriko \( P \).
    \item \( \text{tr}(ABP) = \text{tr}(APB\), ce so A,B,P simetricne matirke. 
    \item \( \text{tr}(ABP) = \text{tr}(A^TB^TP^T)\).
\end{enumerate}

Za poljubna vektorja \( x,y \in \mathbb{R}^n \) velja:
\[
    \text{tr} (xy^T) = \text{tr}(x^Ty)
\]


\textbf{7.7} Ce ima matrika A lastno vrednost $\lambda$, ki ji pripada lastni vektor $\vec{x}$,
potem ima matrika $A + cI$ lastno vrednost $\lambda + c$ z istim lastnim vektorjem $\vec{x}$ (velja samo z
enotskimi matrikami I).

\textbf{7.8} Lastne vrednosti trikotne matrike so enake diagonalnim elementom.

\textbf{7.9} Denimo, da ima matrika $A \in R^{n \times n}\; n$ linearno neodvisnih lastnih vektorjev
$\vec{x}_{1}, \vec{x}_{2}, \dots, \vec{x}_{n}$. Ce jih zlozimo kot stolpce v matriko S
\begin{center}
    \begin{math}
        S =
        \begin{bmatrix}
            \vec{x}_{1}, \vec{x}_{2}, \dots, \vec{x}_{n}
        \end{bmatrix}
    \end{math},
\end{center}
potem je T =: $S^{-1}AS$ diagonalna matrika z lastnimi vrednostmi $\lambda_{i}, i = 1, \dots, n$ na diagonali
\begin{center}
    \begin{math}
        S^{-1}AS = T =
        \begin{bmatrix}
            \lambda_{1} & &\\
            &   \ddots  &  \\
            & &     \lambda_{n} 
        \end{bmatrix}
    \end{math}.
\end{center}

\textbf{Pozor!} Lastni vektorji v matriki S morajo biti v istem vrstnem redu kot lastne vrednosti v matriki $T$.

\textbf{7.10} Ce je $A = STS^{-1}$, potem je $A^{k} = ST^{k}S^{-1}$ za vsak $k \in N$.

\textbf{7.12} Vse lastne vrednosti realne simetricne matrike so realne.

\textbf{7.13} Lastni vektorji realne simetricne matrike, ki pripadajo razlicnim lastnim
vrednostim, so med seboj pravokotni.

\textbf{7.14 Schurov izrek} Za vsako kvadratno matriko reda n, ki ima le realne lastne vrednosti,
obstaja taka ortogonalna matrika $Q$, da je 
\begin{center}
    \begin{math}
        Q^{T}AQ = T
    \end{math}
\end{center} 
zgornjetrikotna matrika, ki ima lastne vrednosti(lahko so kompleksne) matrike A na diagonali.

\textbf{7.15 Spektralni izrek} Vsako simetricno matriko A lahko razcepimo v produkt
$A = QTQ^{T}$, kjer je Q ortogonalna matrika lastnih vektorjev, T pa diagonalna z lastnimi
vrednostmi matrike A na diagonali.

\textbf{7.16} Vsako realno simetricno matriko lahko zapisemo kot linearno kombinacijo matrik ranga 1
\begin{center}
    \begin{math}
        A = \lambda_{1}\vec{q}_{1}\vec{q}_{1}^{T} + \lambda_{2}\vec{q}_{2}\vec{q}_{2}^{T} +
        \dots + \lambda_{n}\vec{q}_{n}\vec{q}_{n}^{T} 
    \end{math},
\end{center}
kjer so $\vec{q}_{i}$ stolpci matrike Q (torej lastni vektorji matrike A).

\textbf{7.17} Za simetricno nesingularno matriko A je stevilo pozitivnih pivotov enako
stevilu pozitivnih lastnih vrednosti.

\textbf{7.18} Kvadratna matrika je pozitivno definirana, kadar so vse njene lastne vrednosti pozitivne.

\textbf{7.19} Kvadratna matrika reda 2 je pozitivno definirana natanko tedaj, kadar sta 
pozitivni sled in determinanta matrike.

\textbf{7.20} Simetricna matrika A reda $n$ je pozitivno definirana natanko tedaj, ko je za vsak
vektor $\vec{x} \neq \vec{0} \in R^{n}$
\begin{center}
    $\vec{x}^{T}A\vec{x} > 0$
\end{center}

\textbf{7.21} Ce sta matriki A in B pozitivno definitni, je pozitivno definitna tudi 
njuna vsota $A + B$.

\textbf{7.22} Matrika A je pozitivno definitna, kadar so vse njene vodilne glavne poddeterminante pozitivne.

\textbf{7.23} Ce so stolpci matrike R linearno neodvisni, je matrika $A = R^{T}R$ pozitivno definitna.

\textbf{7.24} Za vsako simetricno pozitivno definitno matriko A obstaja zgornjetrikotna matrika R, da
je $A = R^{T}R$.

\textbf{7.25} Simetricna matrka reda $n$, ki ima eno od spodnjih lastnosti, ima tudi ostale stiri:
\begin{enumerate}
    \item Vseh $n$ pivotov je pozitivnih;
    \item Vseh $n$ vodilnih glavnih determinant je pozitivnih;
    \item Vseh $n$ lastnih vrednosti je pozitivnih;
    \item Za vsak $\vec{x} \neq \vec{0}$ je $\vec{x}^{T}A\vec{x} > 0$;
    \item $A= R^{T}R$ za neko matriko R z linearno neodvisnimi stolpci.
\end{enumerate}

\textbf{7.26} Vsako realno $m \times n$ matriko A lahko zapisemo kot produkt
$A = UEV^{T}$, kjer je matrika U ortogonalna $m \times m$, E diagonalna $m \times n$ in
V ortogonalna $n \times n$.

\textbf{7.27} Ce je  matrika A simetricna in so vsej njeni elementi realni, potem je njen rang enak stevilu nenicelnih lastnih
vrednosti matrike A.
\begin{center}
    $rang(A)$ = stevilo $\lambda A$
\end{center}

\textbf{7.28 Diagonalizacija} oz \textit{podobnost} matrik. Matriki A in B sta \textit{podobni}, ce imata
obe iste lastne vrednosi. Diagonalno matriko sestavimo tako, da v njeno diagonalo vpisemo lastne vrednosti. Matriko 
P pa sestavimo iz njenih lastnih vektorjev; po stolpcih.
\begin{center}
    \begin{math}
        A = PDP^{-1}
    \end{math} oz.\\
    \begin{math}
        D = P^{-1}AP
    \end{math}
\end{center}

\textbf{7.29 Spektralni razcep}
Naj bodo vekotrji $\vec{q}_{1}, \dots, \vec{q}_{n}$ ONB iz l. vektorjev marike A za l. vrednost $\lambda_{1}, \dots, \lambda{n}$,
potem lahko matriko A zapisemo kot:
\begin{center}
    \begin{math}
        A = \lambda_{1} \vec{q_{1}} \vec{q_{1}}^{T} + \dots + \lambda_{n} \vec{q_{n}} \vec{q_{n}}^{T}
    \end{math}
\end{center}

\textbf{7.30 Nekaj lastnosti simetricnih matrik}
\begin{itemize}
    \item Vse lastne vrednosti simetricne matrike so realne. Lastni vektorji realne simetricne matrike, ki 
    pripadajo razlicnim lastnim vrednostim, so med seboj pravokotni.
    \item Vsako realno simetricno matriko A lahko zapisemo kot $A = QDQ^{T}$, kjer je Q ortogonalna matrika lastnih vektorjev, D pa diagonalna matrika,
    ki ima na diagonali pripadajoce lastne vrednosti matrike A.
\end{itemize}

\section{\underline{Napredna linearna algebra}}

\subsection{Schurov izrek}

\textbf{(Schur)}: Naj bo \( A \in \mathbb{R}^{n \times n} \) matrika z lastnimi vrednostmi \( \lambda_1, \ldots, \lambda_n \). Potem obstaja ortogonalna matrika \( Q \in \mathbb{R}^{n \times n} \) in zgornje trikotna matrika \( Z \), ki ima na diagonali \( \lambda_1, \ldots, \lambda_n \), da velja
\[ A = QZQ^{-1} = QZQ^T. \]

\textbf{Postopek za izračun Schurovega razcepa:}

Firstly, pick an eigenvalue and corresponding eigenvector:
\[
Aq_1 = \lambda_1 q_1 \quad \text{with} \quad q_1^Tq_1=1
\]
Then, find all orthogonal vectors such that you compose a matrix \( Q = [q_1 \dots q_n] \).

Then:
\[
T = Q^TAQ
\]
Compute:
\[
T = \begin{bmatrix}
\lambda_1 & b^T \\
0 & A_2
\end{bmatrix}
\]
(not upper triangular)

Continue with \( A_2 \). Final schur:
\[
Q = \begin{bmatrix}
q_1 & 0 \\
0 & Q_2
\end{bmatrix}
\]
\[
\vdots
\]
\[
Q = \begin{bmatrix}
q_1 & 0 & \dots & 0 \\
0 & q_2 & \dots & 0 \\
\vdots & \vdots & \ddots & \vdots \\
0 & 0 & \dots & q_{n-1}
\end{bmatrix}
\]

We get:
\[
A = QZQ^T
\]
Where \( Z \) is the upper triangular matrix.


\begin{itemize}
    \item \textbf{Posledica:} Vsaka matrika \( A \in \mathbb{R}^{n \times n} \) je podobna zgornje trikotni matriki.
    
    \item \textbf{Posledica:} Vsaka simetrična matrika \( A \in \mathbb{R}^{n \times n} \) je ortogonalno podobna diagonalni matriki.
    
    \item \textbf{Posledica:} Če ima matrika \( A \in \mathbb{R}^{n \times n} \) lastne vrednosti enake \( \lambda_1, \lambda_2, \ldots, \lambda_n \), potem je
    \[
    \text{tr}(A) = \lambda_1 + \lambda_2 + \ldots + \lambda_n
    \]
    in
    \[
    \text{det}(A) = \lambda_1 \lambda_2 \ldots \lambda_n.
    \]
    
    \item \textbf{Posledica (Cayley-Hamilton):} Če je \( \Delta_A(x) = \text{det}(A - xI_n) \) karakteristični polinom matrike \( A \), potem velja \( \Delta_A(A) = 0 \).
\end{itemize}

\subsection{Frobeinusova norma}

Za matriki \( A \in \mathbb{R}^{m \times n} \) in \( B \in \mathbb{R}^{m \times n} \) definiramo
\[
\langle A, B \rangle = \text{tr}(A^T B).
\]

Za produkt \( \langle A, B \rangle: \mathbb{R}^{m \times n} \times \mathbb{R}^{m \times n} \rightarrow \mathbb{R} \) velja za vse matrike \( A, B, C \in \mathbb{R}^{m \times n} \) in za vse \( \alpha, \beta \in \mathbb{R} \),
\begin{enumerate}
    \item \( \langle A, B \rangle = \langle B, A \rangle \),
    \item \( \langle \alpha A + \beta B, C \rangle = \alpha \langle A, C \rangle + \beta \langle B, C \rangle \),
    \item \( \langle A, A \rangle \geq 0 \),
    \item \( \langle A, A \rangle = 0 \) natanko tedaj, ko je \( A = 0 \).
\end{enumerate}
Zato \( \langle A, B \rangle \) imenujemo skalarni produkt matrik \( A \) in \( B \).

Za matrike \( A \in \mathbb{R}^{m \times n} \), \( B \in \mathbb{R}^{m \times k} \) in \( C \in \mathbb{R}^{k \times n} \) velja
\[
\langle A, BC \rangle = \langle B^T A, C \rangle = \langle A C^T, B \rangle.
\]


\textbf{Frobeniusova norma matrike} \( A = [a_{ij}] \in \mathbb{R}^{m \times n} \) je definirana kot
\[
\|A\|_F = \|A\| = \sqrt{\langle A, A \rangle} = \sqrt{\text{tr}(A^T A)}.
\]

Velja:
\[
\|A\|_F^2 = \sum_{i=1}^{n} \sum_{j=1}^{m} a_{ij}^2 = \sum_{i=1}^{\text{min}(m,n)} \sigma_i^2.
\]

Posledica:

\[
\|A\|_F = \sqrt{\sum_{i=1}^{n} \lambda_i^2}
\]

\textbf{(Eckart, Young).} Naj bo \( A = U\Sigma V^T \) razcep singularnih vrednosti matrike \( A \in \mathbb{R}^{m \times n}, m \geq n \), kjer \( U = [u^{(1)} \ldots u^{(m)}] \) in \( \mathbb{R}^{m \times m} \) in \( V = [v^{(1)} \ldots v^{(n)}] \) in \( \mathbb{R}^{n \times n} \). Potem je matrika \( A_k \) iz \( \mathbb{R}^{m \times n} \) ranga \( k \), \( k \leq n \), ki je med vsemi matrikami ranga \( k \) v Frobeniusovi normi najbližje matriki \( A \), enaka
\[
A_k = \sigma_1 u^{(1)}(v^{(1)})^T + \sigma_2 u^{(2)}(v^{(2)})^T + \ldots + \sigma_k u^{(k)}(v^{(k)})^T
\]
in velja
\[
\| A - A_k \|_F = \sqrt{\sigma_{k+1}^2 + \ldots + \sigma_n^2}.
\]
(Velja torej \( \|A - A_k\|_F \leq \|A - X\|_F \) za \( \|A - X\|_F \) za vse matrike \( X \in \mathbb{R}^{m \times n} \), za katere velja \( \text{rank}(X) = k \).)

\subsection{Kroneckerjev produkt}

Kroneckerjev produkt (tudi tenzorski produkt) matrik \( A = [a_{ij}] \in \mathbb{R}^{m \times n} \) in \( B \in \mathbb{R}^{p \times q} \) je \( mp \times nq \) matrika

\begin{math}
A \otimes B = 
\begin{bmatrix}
a_{11}B & a_{12}B & \cdots & a_{1n}B \\
a_{21}B & a_{22}B & \cdots & a_{2n}B \\
\vdots  & \vdots  & \ddots & \vdots  \\
a_{m1}B & a_{m2}B & \cdots & a_{mn}B \\
\end{bmatrix}
\in \mathbb{R}^{mp \times nq}.
\end{math}

Če so matrike $A, B, C$ in $D$ primerne velikosti, potem veljajo naslednje enakosti:
\begin{enumerate}
\item $0 \otimes A = A \otimes 0 = 0$
\item $\alpha \otimes A = A \otimes \alpha = \alpha A$ za vsak $\alpha \in \mathbb{R}$
\item $(\alpha A) \otimes B = A \otimes (\alpha B) = \alpha (A \otimes B)$
\item $(A + B) \otimes C = A \otimes C + B \otimes C$ in $A \otimes (B + C) = A \otimes B + A \otimes C$
\item $(A \otimes B)^T = A^T \otimes B^T$
\item $(A \otimes B) \otimes C = A \otimes (B \otimes C)$.
\item $(A \otimes B)(C \otimes D) = (AC) \otimes (BD)$.
\item $(A \otimes B)^{-1} = A^{-1} \otimes B^{-1}$ če $A$ in $B$ obrnljivi.
\item $\text{tr}(A \otimes B) = \text{tr}(A) \text{tr}(B)$
\item $\text{rang}(A \otimes B) = \text{rang}(A) \text{rang}(B)$
\item Če ima matrika $A \in \mathbb{R}^{n \times n}$ lastne vrednosti $\lambda_1, \ldots, \lambda_m$ in ima matrika $B$ lastne vrednosti $\mu_1, \ldots, \mu_n$, potem je množica lastnih vrednosti matrike $A \otimes B$ enaka:
$$ S_\lambda  = \{ \lambda_i \mu_j; \lambda_i \text{ lastna vrednost } A, \mu_j \text{ lastna vrednost } B\} $$
$$\text{in } |S_\lambda| \leq mn$$
Ravno tako velja potem za lastne vektorje $ v_i \otimes w_j$, da dobimo lastne vektorje matrike $A \otimes B$.
\item Če $A \in \mathbb{R}^{n \times n}$ in $B \in \mathbb{R}^{m \times m}$, potem je $\text{det}(A \otimes B) = (\text{det} A)^m(\text{det} B)^n.$
\end{enumerate}

Posledica:

\[
||A \otimes B||_F = ||A||_F \cdot ||B||_F
\]

\subsection{Kroneckerjeva vsota}

Kroneckerjeva vsota je definirana za kvadratni matriki \( A \) in \( B \):
\[ A \oplus B = A \otimes I_m + I_n \otimes B \]
kjer \( A \in \mathbb{R}^{n \times n} \), \( B \in \mathbb{R}^{m \times m} \).

\(\text{Če so } \lambda_1, \ldots, \lambda_n \) lastne vrednosti \( A \) za lastne vektorje \( u_1, \ldots, u_n \) in \( \mu_1, \ldots, \mu_m \) lastne vrednosti \( B \) za lastne vektorje \( v_1, \ldots, v_n \), potem so
\[ \lambda_i \cdot \mu_j, \quad i = 1, \ldots, n; j = 1, \ldots, m \]
lastne vrednosti za \( A \oplus B \), lastni vektorji pa so
\[ u_i \otimes v_j \]
za \( i \) in \( j \). Lastni vektorji \( A \oplus B \) so enaki \( u_i \otimes v_j \).


\subsection{Vektorizacija}

Za matriko \( A \in \mathbb{R}^{m \times n} \) označimo vektorizacijo matrike \( A \) kot
\[
\text{vec}(A) = \begin{bmatrix}
A^{(1)} \\
A^{(2)} \\
\vdots \\
A^{(n)}
\end{bmatrix} \in \mathbb{R}^{mn}.
\]
vec je preslikava iz \( \mathbb{R}^{m \times n} \) v \( \mathbb{R}^{mn} \).

Za matrike \( A \in \mathbb{R}^{m \times n} \), \( B \in \mathbb{R}^{n \times p} \) in \( C \in \mathbb{R}^{p \times r} \) velja:
\[
\text{vec}(ABC) = (C^T \otimes A)\text{vec}(B).
\]

\subsection{Definitnost matrik}

Spomnimo se, da ima simetrična matrika \( A \in \mathbb{R}^{n \times n} \) vse lastne vrednosti realne.

Simetrični matriki \( A \in \mathbb{R}^{n \times n} \) pravimo
\begin{itemize}
  \item \textbf{pozitivno semidefinitna}, če je \( \mathbf{x}^T A \mathbf{x} \geq 0 \) za vse \( \mathbf{x} \in \mathbb{R}^n \).
  \item \textbf{pozitivno definitna}, če je \( \mathbf{x}^T A \mathbf{x} > 0 \) za vse neničelne \( \mathbf{x} \in \mathbb{R}^n \).
  \item \textbf{negativno semidefinitna}, če je \( \mathbf{x}^T A \mathbf{x} \leq 0 \) za vse \( \mathbf{x} \in \mathbb{R}^n \).
  \item \textbf{negativno definitna}, če je \( \mathbf{x}^T A \mathbf{x} < 0 \) za vse neničelne \( \mathbf{x} \in \mathbb{R}^n \).
  \item \textbf{nedefinitna}, če je \( \mathbf{x}^T A \mathbf{x} > 0 \) za nekatere \( \mathbf{x} \in \mathbb{R}^n \) in \( \mathbf{y}^T A \mathbf{y} < 0 \) za nekatere \( \mathbf{y} \in \mathbb{R}^n \).
\end{itemize}

Posledica: Naj \( A \in \mathbb{R}^{n \times n} \) simetrična z lastnimi vrednostmi \( \lambda_i, \ldots, \lambda_n \).
\begin{itemize}
  \item \( A \) je \textbf{PSD} (pozitivno semidefinitna) \( \Leftrightarrow \lambda_i \geq 0 \) za \( i=1,\ldots,n \).
  \item \( A \) je \textbf{PD} (pozitivno definitna) \( \Leftrightarrow \lambda_i > 0 \) za \( i=1,\ldots,n \).
  \item \( A \) je \textbf{NSD} (negativno semidefinitna) \( \Leftrightarrow \lambda_i \leq 0 \) za \( i=1,\ldots,n \).
  \item \( A \) je \textbf{ND} (negativno definitna) \( \Leftrightarrow \lambda_i < 0 \) za \( i=1,\ldots,n \).
  \item \( A \) je \textbf{nedefinirana} \( \Leftrightarrow \) ima tako pozitivne kot negativne lastne vrednosti.
\end{itemize}
\( A \) je \textbf{PD} \( \Leftrightarrow A \) je \textbf{PSD} in \( A \) obrnljiva.

\textbf{(Sylvester)}. Simetrična matrika \( A \) je pozitivno definitna natanko tedaj, ko so determinante vseh vodilnih glavnih podmatrik matrike \( A \) pozitivne.

\[
\text{det} \left[
\begin{array}{cccc}
+ & + & + & + \\
+ & + & + & + \\
+ & + & + & + \\
+ & + & + & + \\
\end{array}
\right] > 0 \quad \sim \text{PD} \\
\]

Simetrična matrika \( A \) je negativno definitna natanko tedaj, ko je determinanta vsake \( k \times k \) vodilne glavne podmatrike \( A \) pozitivna, če je \( k \) sodo število, ter negativna, če je \( k \) liho število.

\[
\text{det} \left[
\begin{array}{cccc}
- & + & - & + \\
+ & - & + & - \\
- & + & - & + \\
+ & - & + & - \\
\end{array}
\right] \quad \sim \text{ND}
\]

Izrek: Naj \( A \in \mathbb{R}^{n \times n} \) simetrična ranga \( r \). Velja
\begin{itemize}
    \item \( A \) je PSD \( \Leftrightarrow \) obstaja \( B \in \mathbb{R}^{n \times r} \), da je \( A = BB^T \).
    \item \( A \) je PD \( \Leftrightarrow \) obstaja \( B \in \mathbb{R}^{n \times n} \), da je \( A = BB^T \).
    \item \( A \) je NSD \( \Leftrightarrow \) obstaja \( B \in \mathbb{R}^{n \times r} \), da je \( A = -BB^T \).
    \item \( A \) je ND \( \Leftrightarrow \) obstaja \( B \in \mathbb{R}^{n \times n} \), da je \( A = -BB^T \).
    \item \( A \) je nedefinirana \( \Leftrightarrow \) obstaja tako pozitivne kot negativne lastne vrednosti.
\end{itemize}
\( A \) je PD \( \Leftrightarrow A \) je PSD in \( A \) obrnljiva.

\textbf{(Razcep Choleskega)}. Obrnljiva matrika \( A \in \mathbb{R}^{n \times n} \) ima razcep Choleskega
\[ A = LL^T, \]
kjer je \( L \in \mathbb{R}^{n \times n} \) spodnje trikotna matrika, natanko tedaj, ko je \( A \) simetrična in pozitivno definitna.

Z uporabo spodnjega (rekurzivnega) algoritma:
Simetrično matriko \( A \in \mathbb{R}^{n \times n} \) zapišemo v bločni obliki

\[
A_1 := A = \begin{bmatrix}
a_{11} & b^T \\
b & B
\end{bmatrix}
\]

in definiramo

\[
L_1 := \begin{bmatrix}
\sqrt{a_{11}} & 0^T \\
\frac{1}{\sqrt{a_{11}}} b & I_{n-1}
\end{bmatrix}.
\]

Tedaj je

\[
A_1 = \begin{bmatrix}
a_{11} & b^T \\
b & B
\end{bmatrix} = L_1 \begin{bmatrix}
1 & 0^T \\
0 & B - \frac{1}{a_{11}} bb^T
\end{bmatrix} L_1^T.
\]

Ponovimo na simetrični matriki \( A_2 := B - \frac{1}{a_{11}} bb^T \in \mathbb{R}^{(n-1) \times (n-1)} \).

Naj bodo \( L_2, L_3, \ldots, L_n \) matrike, ki jih dobimo v ponovljenih korakih. Matrika \( L \) je potem

\[
L = L_1 \cdot \left[ \begin{array}{cc}
1 & 0^T \\
0 & L_2
\end{array} \right] \cdot \left[ \begin{array}{cc}
I_2 & 0 \\
0 & L_3
\end{array} \right] \cdot \ldots \cdot \left[ \begin{array}{cc}
I_{n-1} & 0 \\
0 & L_n
\end{array} \right]
\]

\subsection{Vektorski prostori}

\textbf{Realni vektorski prostor} \( V \) je množica \textbf{vektorjev} \( v \), za katere imamo definirani dve notranji operaciji
\begin{itemize}
    \item seštevanje vektorjev \( (u, v \in V \Rightarrow u+v \in V) \),
    \item množenje vektorjev z realnimi števili \( (v \in V, \alpha \in \mathbb{R} \Rightarrow \alpha v = \alpha \cdot v \in V) \),
\end{itemize}
z lastnostmi
\begin{enumerate}
    \item \( u + v = v + u \) in \( (u + v) + w = u + (v + w) \),
    \item obstaja ničelni vektor \( 0 \) in velja \( v + 0 = 0 + v = v \),
    \item za vsak \( v \in V \) obstaja nasprotni vektor \( -v \), za katerega velja \( v + (-v) = (-v) + v = 0 \),
    \item \( 1 \cdot v = v \) za vsak \( v \in V \),
    \item \( (\alpha\beta) \cdot v = \alpha \cdot (\beta \cdot v) \),
    \item \( (\alpha + \beta) \cdot v = \alpha \cdot v + \beta \cdot v \),
    \item \( \alpha \cdot (u + v) = \alpha \cdot u + \alpha \cdot v \),
\end{enumerate}
za poljubne \( u, v, w \in V \) in \( \alpha, \beta \in \mathbb{R} \).


\textbf{Izrek:} Naj bo \( V \) vektorski prostor. Potem velja
\begin{enumerate}
    \item \( V \) vsebuje ničelni vektor \( 0 \),
    \item v vsakem vektorskem prostoru \( V \) je ničelni vektor \( 0 \) en sam,
    \item \( \alpha \cdot 0 = 0 \) za vsak \( \alpha \in \mathbb{R} \),
    \item \( 0 \cdot v = 0 \) za vsak \( v \in V \).
\end{enumerate}
Za vektorje \( v_1, v_2, \ldots, v_n \in V \) in skalare \( \alpha_1, \alpha_2, \ldots, \alpha_n \in \mathbb{R} \) imenujemo vektor
\[ \alpha_1v_1 + \alpha_2v_2 + \ldots + \alpha_nv_n \]
\textbf{linearna kombinacija vektorjev} \( v_1, v_2, \ldots, v_n \).
Denimo, ničelni vektor \( 0 \) je
\[ 0 \cdot v_1 + 0 \cdot v_2 + \ldots + 0 \cdot v_n \]
je linearna kombinacija poljubnih vektorjev \( v_1, v_2, \ldots, v_n \in V \). Linearno kombinacijo z izključno ničelnimi koeficienti imenujemo \textbf{trivialna linearna kombinacija}.

Če je podmnožica \( U \) vektorskega prostora \( V \)
\begin{itemize}
    \item[(1)] zaprta za seštevanje \( (u, v \in U \Rightarrow u + v \in U) \) in
    \item[(2)] zaprta za množenje vektorjev z realnimi števili \( (v \in U, \alpha \in \mathbb{R} \Rightarrow \alpha v \in U) \),
\end{itemize}
potem jo imenujemo \textbf{vektorski podprostor} prostora \( V \).

\textbf{Izrek:} Podmnožica \( U \) vektorskega prostora \( V \) je vektorski podprostor natanko tedaj, ko je poljubna linearna kombinacija \( \alpha u + \beta v \) vektorjev \( u, v \in U \) tudi vsebovana v \( U \).

Vsak vektorski podprostor po (2) vsebuje tudi vektor \( 0 \cdot v = 0 \). Zatorej podmnožica vektorskega prostora, ki ne vsebuje ničelnega vektorja, ne more biti vektorski podprostor.

Ker lastnosti (1)-(7) veljajo za poljubne elemente vektorskega prostora \( V \), veljajo tudi za vse elemente vektorskega podprostora \( U \) v \( V \). Poleg tega je vektorski podprostor po definiciji zaprt za seštevanje in množenje s števili. Zatorej je vsak vektorski podprostor hkrati tudi vektorski prostor.

\subsubsection{Linearna ogrinjača} \( \mathcal{L}\{v_1, v_2, \ldots, v_n\} \) vektorjev \( v_1, v_2, \ldots, v_n \) je množica vseh linearnih kombinacij vektorjev \( v_1, v_2, \ldots, v_n \).

Ker je linearna kombinacija linearnih kombinacij vektorjev \( v_1, v_2, \ldots, v_n \in V \) zopet linearna kombinacija vektorjev \( v_1, v_2, \ldots, v_n \), je po Izreku 2 linearna ogrinjača \( \mathcal{L}\{v_1, v_2, \ldots, v_n\} \) linearni podprostor v \( V \). Pravimo, da vektorji \( v_1, v_2, \ldots, v_n \) \textbf{napenjajo prostor} \( \mathcal{L}\{v_1, v_2, \ldots, v_n\} \).

Ne le, da je linearna ogrinjača vektorski prostor. Velja celo več.

 Linearna ogrinjača vektorjev \( v_1, v_2, \ldots, v_n \), vektorskega prostora \( V \) je najmanjši vektorski podprostor v \( V \), ki vsebuje vektorje \( v_1, v_2, \ldots, v_n \).

\subsubsection{Baza vektorskega prostora}

Vektorji \( v_1, v_2, \ldots, v_n \) v \( V \) so \textbf{linearno odvisni}, če obstaja vektor \( v_k \), ki je linearna kombinacija ostalih \( v_1, v_2, \ldots, v_{k-1}, v_{k+1}, \ldots, v_n \):
\[ v_k = \alpha_1v_1 + \alpha_2v_2 + \ldots + \alpha_{k-1}v_{k-1} + \alpha_{k+1}v_{k+1} + \ldots + \alpha_nv_n, \]
kjer \( \alpha_i \in \mathbb{R} \).

Vektorji \( v_1, v_2, \ldots, v_n \) v \( V \) so \textbf{linearno neodvisni}, če niso linearno odvisni. Ekvivalentno, \( v_1, v_2, \ldots, v_n \) v \( V \) so linearno neodvisni, če je njihova trivialna linearna kombinacija edina njihova linearna kombinacija, ki je enaka ničelnemu vektorju \( 0 \). Z drugimi besedami, \( v_1, v_2, \ldots, v_n \) v \( V \) so linearno neodvisni, če iz
\[ \alpha_1v_1 + \alpha_2v_2 + \ldots + \alpha_nv_n = 0 \]
sledi
\[ \alpha_1 = \alpha_2 = \ldots = \alpha_n = 0. \]


Množica vektorjev \( \{v_1, v_2, \ldots, v_n\} \) je \textbf{baza} vektorskega prostora \( V \), če 
\begin{enumerate}
    \item[(B1)] so \( v_1, v_2, \ldots, v_n \) linearno neodvisni in
    \item[(B2)] \( v_1, v_2, \ldots, v_n \) napenjajo prostor \( V \).
\end{enumerate}

\textbf{Izrek:} Vsak vektorski prostor ima neštevno baz. Vse baze vektorskega prostora imajo enako število vektorjev.

\emph{Dimenzija prostora} \( V \) je enaka moči (poljubne) baze prostora \( V \). Označimo jo z \(\dim V\).

\textbf{Izrek:} Za vsako bazo vektorskega prostora \( V \) je zapis poljubnega vektorja \( v \in V \) kot linearna kombinacija baznih vektorjev vedno enoličen.


\subsubsection{Linearne preslikave}
Naj bosta \( V \) in \( U \) vektorska prostora. Preslikava \( \tau: V \to U \) je \textbf{linearna preslikava}, če velja
\begin{itemize}
    \item[(1)] \( \tau(v + u) = \tau(v) + \tau(u) \) za vsaka \( v, u \in V \) in
    \item[(2)] \( \tau(\alpha v) = \alpha \tau(v) \) za vsak \( v \in V \) in vsak \( \alpha \in \mathbb{R} \).
\end{itemize}

Preslikava \( \tau: V \to U \) je linearna natanko tedaj, ko velja
\[ \tau(\alpha v + \beta u) = \alpha \tau(v) + \beta \tau(u) \]
za vse \( v, u \in V \) ter vse \( \alpha, \beta \in \mathbb{R} \).

Za poljubno linearno preslikavo \( \tau: V \to U \) velja \( \tau(0_V) = 0_U \).

Naj bodo \( \tau, \psi: V \to U \) ter \( \theta: U \to W \) linearne preslikave in naj bo \( \gamma \in \mathbb{R} \).
\begin{enumerate}
    \item[(1)] \textbf{Vsota} \( \tau + \psi: V \to U \) je preslikava, definirana s predpisom
    \[ (\tau + \psi)(v) = \tau(v) + \psi(v). \]
    \item[(2)] \textbf{Produkt s skalarjem} \( \gamma\tau: V \to U \) je preslikava, definirana s predpisom
    \[ (\gamma\tau)(v) = \gamma \tau(v). \]
    \item[(3)] \textbf{Kompozitum} \( \theta \circ \tau: V \to W \) je preslikava, definirana s predpisom
    \[ (\theta \circ \tau)(v) = \theta(\tau(v)). \]
\end{enumerate}

\textbf{Izrek:} Vsota, produkt s skalarjem in kompozitum linearnih preslikav so linearne preslikave.

\textbf{Posledica:} Množica vseh linearnih preslikav iz vektorskega prostora \( V \) v vektorski prostor \( U \) je vektorski prostor

\textbf{Izrek:} Naj bodo \( \tau, \psi: V \to U \) ter \( \theta: U \to W \) linearne preslikave in naj bo \( \alpha \in \mathbb{R} \).
\begin{enumerate}
    \item Matrika, ki ustreza vsoti preslikav \( \tau + \psi \), je enaka vsoti matrik posameznih preslikav.
    \[ A_{\tau+\psi,B}^C = A_{\tau,B}^C + A_{\psi,B}^C \]
    \item Matrika, ki ustreza produktu s skalarjem \( \alpha\tau \), je enaka večkratniku matrike preslikave.
    \[ A_{\alpha\tau,B}^C = \alpha A_{\tau,B}^C \]
    \item Matrika, ki ustreza kompozitumu preslikav, je enaka produktu matrik posameznih preslikav.
    \[ A_{\theta\circ\tau,B}^D = A_{\theta,C}^D \cdot A_{\tau,B}^C \]
    \item Matrika, ki ustreza inverzu obrnljive preslikave, je enaka inverzu matrike te preslikave. Torej, če je \( \tau \) obrnljiva preslikava, je obrnljiva tudi matrika \( A_{\tau,B}^C \). Velja
    \[ A_{\tau^{-1},C}^B = (A_{\tau,B}^C)^{-1} \].
\end{enumerate}


Neničelnemu vektorju \( v \) v \( V \) pravimo \textit{lastni vektor} linearne preslikave \( \tau: V \to V \), če velja
\[ \tau(v) = \lambda v. \]
Številu \( \lambda \) pravimo \textit{lastna vrednost} linearne preslikave \( \tau \).

\textbf{Izrek:} Vsaka lastna vrednost linearne preslikave \( \tau \) je tudi lastna vrednost poljubne matrike \( A_{\tau} \), ki pripada preslikavi \( \tau \). Vse matrike, ki pripadajo dani linearni preslikavi \( \tau \) imajo enake lastne vrednosti.

Pravimo, da je linearno preslikavo \( \tau: V \to V \) mogoče \textit{diagonalizirati}, če obstaja baza, v kateri pripada preslikavi diagonalna matrika.

\textbf{Izrek:} Linearno preslikavo \( \tau: V \to V \) je mogoče diagonalizirati natanko tedaj, ko obstaja baza prostora \( V \) sestavljena iz lastnih vektorjev preslikave \( \tau \).

Naj bo \( \tau: V \rightarrow U \) linearna preslikava vektorskega prostora \( V \) v vektorski prostor \( U \).

\textbf{Def:} Jedro linearne preslikave \( \tau \) je množica \( \ker(\tau) \) vseh vektorjev \( v \in V \), za katere velja
\[ \tau(v) = 0. \]

\textbf{Def:} Slika linearne preslikave je množica \( \text{im}(\tau) = \{ \tau(v) : v \in V \} \subseteq U \).

\textbf{Izrek:} Jedro \( \ker \tau \) linearne preslikave \( \tau: V \rightarrow U \) je vektorski podprostor v \( V \), slika \( \text{im} \tau \) pa vektorski podprostor v \( U \).

\textbf{Izrek:} Naj bo \( \tau: V \rightarrow U \) linearna preslikava iz vektorskega prostora \( V \) v vektorski prostor \( U \).
\begin{enumerate}
    \item \( \tau \) je injektivna natanko tedaj, ko je \( \ker \tau = \{0\} \).
    \item \( \tau \) je surjektivna natanko tedaj, ko je \( \text{im} \tau = U \).
\end{enumerate}

\textbf{Izrek:} Naj bo \( \tau: V \rightarrow U \) linearna preslikava in naj bo \( A = A_{\tau,B,C} \) matrika, ki pripada preslikavi \( \tau \). Potem je
\begin{enumerate}
    \item \( \text{dim}(\text{im}(\tau)) = \text{rank}(A) \),
    \item \( \text{dim}(\text{ker}(\tau)) + \text{dim}(\text{im}(\tau)) = \text{dim}(V) \).
\end{enumerate}

\textbf{Posledica:} Naj bo \( \tau: V \rightarrow U \) linearna preslikava, \(\text{dim} V = \text{dim} U = n\) in naj bo \( A \) neka matrika, ki pripada \( \tau \). Naslednje trditve so ekvivalentne:
\begin{enumerate}
    \item \( \tau \) je bijektivna.
    \item \( \tau \) je injektivna.
    \item \( \tau \) je surjektivna.
    \item \( A \) je obrnljiva.
    \item \( \text{ker} \tau = \{0\} \).
    \item \( N(A) = \{0\} \).
    \item \( \text{im} \tau = U \).
    \item \( C(A) = \mathbb{R}^n \).
    \item Rang matrike \( A \) je \( n \).
    \item Vrstice matrike \( A \) so linearno neodvisne.
    \item Vrstice matrike \( A \) razpenjajo \( \mathbb{R}^n \).
    \item Vrstice matrike \( A \) tvorijo bazo \( \mathbb{R}^n \).
    \item Stolpci matrike \( A \) so linearno neodvisni.
    \item Stolpci matrike \( A \) razpenjajo \( \mathbb{R}^n \).
    \item Stolpci matrike \( A \) tvorijo bazo \( \mathbb{R}^n \).
    \item \( \det A \neq 0 \).
    \item Homogeni sistem enačb \( Ax = 0 \) ima le trivialno rešitev.
    \item Sistem enačb \( Ax = b \) ima rešitev za vsak \( b \in \mathbb{R}^n \).
\end{enumerate}

\section{\underline{Analiza}}

\subsection{Funkcije več spremenljivk}
Funkcija več spremenljivk

\[
    f : D_f \subseteq \mathbb{R}^n \to \mathbb{R},
\]

kjer

\[
    \mathbf{x} = (x_1, x_2, \ldots, x_n) \mapsto f(x_1, x_2, \ldots, x_n)
\]

je funkcija, ki predpiše realno vrednost $f(\mathbf{x}) = f(x_1, x_2, \ldots, x_n) \in \mathbb{R}$ vsaki točki $\mathbf{x} = (x_1, x_2, \ldots, x_n) \in D_f \subseteq \mathbb{R}^n$. Množici $D_f$ pravimo \textbf{definicijsko območje} funkcije $f$.

V primeru, ko je $n = 2$, je graf funkcije $f = f(x, y): D_f \subseteq \mathbb{R}^2 \to \mathbb{R}$ ploskev v $\mathbb{R}^3$.

\[
    \Gamma_f = \{ (x, y, f(x, y)) : (x, y) \in D_f \}
\]

\textbf{Nivojska krivulja} (ali nivojnica) funkcije $f = f(x, y)$ je množica vseh točk $(x, y) \in D_f$, za katere velja $f(x, y) = c$ za dano realno število $c \in \mathbb{R}$. Tako vsaka točka $(x, y) \in D_f$ leži na natanko eni nivojski krivulji in zato se definicijsko območje $D_f$ razsloji na nivojske krivulje.

\subsubsection{Parcialni odvod}
Parcialni odvod funkcije \( f: \mathbb{R}^n \rightarrow \mathbb{R} \) v točki a = \( (a_1, a_2, \ldots, a_n) \) po spremenljivki \( x_i \) definiramo kot
\[
f_{x_i}(a) = \frac{\partial f}{\partial x_i}(a) = \lim_{h \to 0} \frac{f(a_1, \ldots, a_{i-1}, a_i + h, a_{i+1}, \ldots, a_n) - f(a)}{h}.
\]
Tako nam torej parcialni odvod funkcije \( f \) po \( x_i \), v točki a = \( (a_1, a_2, \ldots, a_n) \) pove relativno spremembo funkcisjke vrednosti pri zelo majhni spremembi spremenljivke \( x_i \), kjer so ostale spremenljivke fiksne.


\subsubsection{Gradient funkcije}

\textbf{Vektor}
\[
(\nabla f)(a) = (f_{x_1}(a), f_{x_2}(a), \ldots, f_{x_n}(a))
\]
imenjujemo \textbf{gradient} funkcije \( f \) v točki \( a \).

\textbf{Smerni odvod funkcije} \( f: \mathbb{R}^n \rightarrow \mathbb{R} \) v točki \( a = (a_1, a_2, \ldots, a_n) \) \textbf{v smeri vektorja} \( \vec{e} \) je enak
\[
f_{\vec{e}}(a) = (\nabla f)(a) \cdot \frac{\vec{e}}{\| \vec{e} \|} = \sum_{i=1}^n \frac{f_{x_i}(a) e_i}{\| \vec{e} \|}
\]

Za funkcijo \( f: \mathbb{R}^n \rightarrow \mathbb{R} \) velja:
\begin{enumerate}
\item Gradient funkcije \( f \) v točki \( a \) kaže v smeri najhitrejšega naraščanja funkcije \( f \) v točki \( a \).
\item V primeru \( n = 2 \) je gradient funkcije \( f = f(x, y) \) v točki \( a \) pravokoten na nivojsko krivuljo v tej točki.
\item Smerni odvod \( f_{\vec{e}}(a) \) je relativna sprememba funkcisjke vrednosti \( f(a) \) ob majhnem premiku iz točke \( a \) v smeri vektorja \( \vec{e} \). Zato velja:
\begin{itemize}
\item Če je \( f_{\vec{e}}(a) > 0 \), potem \( f \) ob majhnem pomiku iz točke \( a \) v smeri vektorja \( \vec{e} \) narašča.
\item Če je \( f_{\vec{e}}(a) < 0 \), potem \( f \) ob majhnem pomiku iz točke \( a \) v smeri vektorja \( \vec{e} \) pada.
\end{itemize}
\end{enumerate}

\subsubsection{Linearna aproksimacija}
Za dano funkcijo \( f: \mathbb{R}^n \rightarrow \mathbb{R} \) lahko v točki a + h blizu a njeno funkcijso vrednost ocenimo s formulo
\[
f(a + h) \approx f(a) + (\nabla f(a)) \cdot h.
\]


\subsubsection{Visji odvodi}
Parcialne odvode drugega reda izračunamo s parcialnim odvajanjem parcialnih odvodov prvega reda. Definiramo jih kot
\[
f_{x_i x_j}(x) = \frac{\partial^2 f}{\partial x_j \partial x_i}(x) = \frac{\partial}{\partial x_j} \left( \frac{\partial f}{\partial x_i}(x) \right).
\]
\( n \times n \) matriko
\[
H_f(x) = \left[ \frac{\partial^2 f}{\partial x_i \partial x_j}(x) \right]_{i,j=1,...,n}
\]
imenujemo \textit{Hessejeva matrika} funkcije \( f \) v točki \( x \). Če sta pri tem \( f_{x_i x_j}, f_{x_j x_i} \) zvezni funkciji, potem sta omenjena druga parcialna odvoda enaka. Zato je v primeru, ko so vsi parcialni odvodi \( f_{x_i x_j} \) zvezni, Hessejeva matrika \( H_f(x, y) \) simetrična matrika.

Pravila: 
\begin{enumerate}
\item \(\frac{\partial \mathbf{x}}{\partial \mathbf{x}} = I_n\)
\item Če je \(A \in \mathbb{R}^{m \times n}\), potem \(\frac{\partial (A\mathbf{x})}{\partial \mathbf{x}} = A\).
\item Če je \(\mathbf{a} \in \mathbb{R}^n\), potem \(\frac{\partial (\mathbf{a}^T \mathbf{x})}{\partial \mathbf{x}} = \mathbf{a}^T\).
\item Če je \(A \in \mathbb{R}^{n \times n}\), potem \(\frac{\partial (\mathbf{x}^T A \mathbf{x})}{\partial \mathbf{x}} = \mathbf{x}^T(A + A^T)\).
\item Če je \(A \in \mathbb{R}^{n \times n}\) simetrična matrika, potem velja \(\frac{\partial (\mathbf{x}^T A \mathbf{x})}{\partial \mathbf{x}} = 2A^T\mathbf{x}\).
\item \(\frac{\partial^2 \mathbf{x}}{\partial \mathbf{x}^2} = 2I_n\).
\item Če \(G: D_G \subseteq \mathbb{R}^m \rightarrow \mathbb{R}^n\) in \(F: D_F \subseteq \mathbb{R}^n \rightarrow \mathbb{R}^p\) in \( \mathbf{H} = F \circ G\), potem \(\frac{\partial \mathbf{H}}{\partial \mathbf{x}} = \frac{\partial F}{\partial \mathbf{G}}(\mathbf{G}(\mathbf{x})) \cdot \frac{\partial \mathbf{G}}{\partial \mathbf{x}}\).
\end{enumerate}


\subsubsection{Vektorska funkcija}
Za vektorsko funkcijo 
\[ F: D_F \subseteq \mathbb{R}^n \rightarrow \mathbb{R}^m, \]
kjer je
\[ \mathbf{x} \mapsto \mathbf{F}(\mathbf{x}) = [f_1(\mathbf{x}), f_2(\mathbf{x}), \ldots, f_m(\mathbf{x})]^T \]
je \( m \)-terica funkcij več spremenljivk.

\subsubsection{Jacobijeva matrika}
Jacobijeva matrika vektorske funkcije 
\[ F: D_F \subseteq \mathbb{R}^n \rightarrow \mathbb{R}^m \]
je \( m \times n \) matrika prvih odvodov funkcij \( f_1, \ldots, f_m \):
\[ 
J_F(\mathbf{x}) = \begin{bmatrix}
\frac{\partial f_1}{\partial x_1}(\mathbf{x}) & \frac{\partial f_1}{\partial x_2}(\mathbf{x}) & \ldots & \frac{\partial f_1}{\partial x_n}(\mathbf{x}) \\
\frac{\partial f_2}{\partial x_1}(\mathbf{x}) & \frac{\partial f_2}{\partial x_2}(\mathbf{x}) & \ldots & \frac{\partial f_2}{\partial x_n}(\mathbf{x}) \\
\vdots & \vdots & \ddots & \vdots \\
\frac{\partial f_m}{\partial x_1}(\mathbf{x}) & \frac{\partial f_m}{\partial x_2}(\mathbf{x}) & \ldots & \frac{\partial f_m}{\partial x_n}(\mathbf{x})
\end{bmatrix}
\]
Absolutna vrednost determinante Jacobijeve matrike vektorske funkcije 
\[ F: D_F \subseteq \mathbb{R}^n \rightarrow \mathbb{R}^m \]
pove, za kakšen faktor funkcija lokalno raztegne prostor.

\subsection{Večkratni integrali}

\subsubsection{Izrek (Fubini, 1)} 
Če je \( f: R \rightarrow \mathbb{R} \) zvezna funkcija na pravokotniku \( R = [a, b] \times [c, d] \subseteq \mathbb{R}^2 \), potem
\[
\iint_R f(x,y) \,dx\,dy = \int_c^d \left( \int_a^b f(x,y) \,dx \right) dy
\]
\[
= \int_a^b \left( \int_c^d f(x,y) \,dy \right) dx.
\]


\subsubsection{Dvojni integrali}

Če je \( D \subseteq \mathbb{R}^2 \) neko omejeno območje in če \( f: D \rightarrow \mathbb{R} \) zvezna funkcija, izberimo tak pravokotnik \( R \), da velja \( D \subseteq R \). Sedaj definiramo dvojni integral funkcije \( f \) na območju \( D \) kot
\[
\iint_D f(x,y) \,dx\,dy = \iint_R F(x,y) \,dx\,dy,
\]
kjer
\[
F(x,y) = \begin{cases} 
f(x,y), & (x,y) \in D \\
0, & (x,y) \not\in D.
\end{cases}
\]

\subsubsection{Izrek (Fubini, 2)} 
\begin{enumerate}
\item Če je \( D = \{(x,y); a \leq x \leq b \text{ in } \varphi_1(x) \leq y \leq \varphi_2(x)\} \) \( \subseteq \mathbb{R}^2 \) in \( f: D \rightarrow \mathbb{R} \) zvezna funkcija, potem je
\[
\iint_D f(x,y) \,dx\,dy = \int_a^b \left( \int_{\varphi_1(x)}^{\varphi_2(x)} f(x,y) \,dy \right) dx.
\]

\item Če je \( D = \{(x,y); \varphi_1(y) \leq x \leq \varphi_2(y) \text{ in } c \leq y \leq d\} \) \( \subseteq \mathbb{R}^2 \) in \( f: D \rightarrow \mathbb{R} \) zvezna funkcija, potem je
\[
\iint_D f(x,y) \,dx\,dy = \int_c^d \left( \int_{\varphi_1(y)}^{\varphi_2(y)} f(x,y) \,dx \right) dy.
\]
\end{enumerate}

\subsubsection{Izrek o menjavi spremenljivk} 
Naj bo \( f: D \rightarrow \mathbb{R} \) zvezna funkcija na \( D \subseteq \mathbb{R}^2 \). Če je \( x = \varphi(u, v) \), \( y = \psi(u, v) \), takašna menjava spremenljivk, da je \( \det J_{\varphi,\psi} \neq 0 \), potem
\[
\iint_D f(x, y) \,dx\,dy = \iint_{D'} f(\varphi(u, v), \psi(u, v)) \left| \det J_{\varphi,\psi} \right| \,du\,dv.
\]

Podobno, če je \( f: D \rightarrow \mathbb{R} \) zvezna funkcija na \( D \subseteq \mathbb{R}^3 \) ter \( x = \varphi(u, v, w) \), \( y = \psi(u, v, w) \), \( z = \chi(u, v, w) \), takašna menjava spremenljivk, da je \( \det J_{\varphi,\psi,\chi} \neq 0 \), potem velja
\[
\iiint_D f(x, y, z) \,dx\,dy\,dz = 
\]
\[
 = \iiint_{D'} f(\varphi(u, v, w), \psi(u, v, w), \chi(u, v, w)) \left| \det J_{\varphi,\psi,\chi} \right| \,du\,dv\,dw.
\]

\subsubsection{Primeri menjave spremenljivk}

\begin{enumerate}
\item \textbf{Polarne koordinate} v \( \mathbb{R}^2 \) so podane z
\[ 
x = r \cos \varphi, \quad y = r \sin \varphi, 
\]
\[ 
r \geq 0, \quad \varphi \in [0, 2\pi], \quad \text{in velja} \quad |\det J_{\text{polar}}| = r.
\]

\item \textbf{Cilindrične koordinate} v \( \mathbb{R}^3 \) so podane z
\[ 
x = r \cos \varphi, \quad y = r \sin \varphi, \quad z = z, 
\]
\[ 
r > 0, \quad \varphi \in [0, 2\pi], \quad z \in \mathbb{R}, \quad \text{in velja} \quad |\det J_{\text{cylindrical}}| = r.
\]

\item \textbf{Sferične koordinate} v \( \mathbb{R}^3 \) so podane z
\[ 
x = r \cos \varphi \cos \theta, \quad y = r \sin \varphi \cos \theta, \quad z = r \sin \theta,
\]
\[ 
r > 0, \quad \varphi \in [0, 2\pi], \quad \theta \in \left[-\frac{\pi}{2}, \frac{\pi}{2}\right], 
\]
\[
\quad \text{in velja} \quad |\det J_{\text{spherical}}| = r^2 \cos \theta.
\]
\end{enumerate}


\subsection{Optimizacija}

\subsection{Klasifikacija Lokalnih ekstremov}

Naj bo \( f: \mathbb{R}^n \rightarrow \mathbb{R} \) ter a v definicijskem območju funkcije \( f \).
Če za vse točke \( x \neq a \), ki so "dovolj blizu" točke a (tj. \( \| x - a \| < \varepsilon \) za nek dovolj majhen \( \varepsilon \)) velja \( f(x) < f(a) \), 
potem pravimo, da ima funkcija \( f \) v točki a \textbf{lokalni maksimum}.

Če za vse točke \( x \neq a \), ki so "dovolj blizu" točke a (tj. \( \| x - a \| < \varepsilon \) za nek dovolj majhen \( \varepsilon \)) velja \( f(x) > f(a) \), 
potem pravimo, da ima funkcija \( f \) v točki a \textbf{lokalni minimum}.

Če je funkcija \( f \) zvezno parcialno odvedljiva, potem je jasno, da ima lahko lokalne ekstreme le v stacionarnih točkah. Torej je potreben pogoj za lokalni ekstrem funkcije \( f \) v točki a:

\[ (\nabla f)(a) = 0, \]

kar pomeni, da moramo lokalne ekstreme iskati zgolj med stacionarnimi točkami.

\subsubsection{Izrek}
Naj bo \( a \) stacionarna točka dvakrat parcialno zvezno odvedljive funkcije \( f: \mathbb{R}^n \rightarrow \mathbb{R} \).

\begin{enumerate}
    \item Če so vse lastne vrednosti matrike \( H_f(a) \) pozitivne, ima \( f \) v \( a \) lokalni minimum.
    \item Če so vse lastne vrednosti matrike \( H_f(a) \) negativne, ima \( f \) v \( a \) lokalni maksimum.
    \item Če so vse lastne vrednosti matrike \( H_f(a) \) neničelne, vendar različno predznačene, lokalnega ekstrema v \( a \) ni.
    \item Če je kakšna od lastnih vrednosti matrike \( H_f(a) \) enaka 0, o lokalnih ekstremih funkcije \( f \) v točki \( a \) iz matrike \( H_f(a) \) ne moremo sklepati.
\end{enumerate}


\subsubsection{Lokalni ekstremi z omejitvami}

Pogosto naletimo na problem iskanja ekstremalnih vrednosti funkcije \( f: \mathbb{R}^n \rightarrow \mathbb{R} \) pri pogojih

\[ g_1(x) = g_2(x) = \ldots = g_m(x) = 0. \]

Izkaže se, da lahko lokalni ekstremi funkcije \( f \) pri pogoju \( g_i(x) = 0, i = 1, \ldots, m, \) nastopijo le v stacionarnih točkah funkcije

\[ L = f - \lambda_1 g_1 - \ldots - \lambda_m g_m, \]

ki je funkcija \( n + m \) spremenljivk \( x_1, x_2, \ldots, x_n, \lambda_1, \lambda_2, \ldots, \lambda_m \).\\
Funkciji \( L \) pravimo \textbf{Lagrangeova funkcija}, novim spremenljivkam \( \lambda_1, \lambda_2, \ldots, \lambda_m \) pa \textbf{Lagrangevi multiplikatorji}.\\
Omenjeni pogoj ni zadosten. Nekatere kritične točke funkcije \( L \) so ekstremalne točke funkcije \( f: \mathbb{R}^n \rightarrow \mathbb{R} \) pod pogoji \( g_1(x) = g_2(x) = \ldots = g_m(x) = 0 \), ostale pa ne.


\subsubsection{Odvodi vektorskih funkcij}
Naj bo \( F: \mathbb{R}^n \rightarrow \mathbb{R}^m, F(x) = 
\begin{bmatrix}
    f_1(x) \\
    f_2(x) \\
    \vdots \\
    f_m(x)
\end{bmatrix} \)
vektorska funkcija na spremenljivk x = \( (x_1, ..., x_n) \).

Spomnimo se, da je odvod vektorske funkcije \( F \) po vektorju spremenljivk \( \tilde{x} \) definiran kot

\[ 
\frac{\partial \tilde{F}}{\partial \tilde{x}} = J_F(\tilde{x}) = 
\begin{bmatrix}
    \frac{\partial f_1}{\partial x_1} (\tilde{x}) & \cdots & \frac{\partial f_1}{\partial x_n} (\tilde{x}) \\
    \vdots & \ddots & \vdots \\
    \frac{\partial f_m}{\partial x_1} (\tilde{x}) & \cdots & \frac{\partial f_m}{\partial x_n} (\tilde{x})
\end{bmatrix}
\]

Drugi odvod funkcije \( f: \mathbb{R}^n \rightarrow \mathbb{R} \) (tu m = 1) pa kot

\[ 
\frac{\partial^2 f}{\partial \tilde{x}^2} = \frac{\partial}{\partial \tilde{x}} \left( \frac{\partial f}{\partial \tilde{x}} \right)^T 
\].

Funkcija \( f: \mathbb{R}^n \rightarrow \mathbb{R} \) je konveksna na \( D \), če velja

\[ 
f(t \tilde{x} + (1-t) \tilde{y}) \leq t f(\tilde{x}) + (1-t) f(\tilde{y}) 
\]

za vse \( \tilde{x}, \tilde{y} \in D \) in za vse \( t \in [0, 1] \). Funkcija \( f \) je konkavna na \( D \), če je funkcija \( -f \) konveksna na \( D \).

\subsubsection{Pravila za odvajanje vektorskih funkcij}

\begin{enumerate}
    \item \(\frac{\partial \tilde{x}}{\partial \tilde{x}} = I_n\)
    \item Če je \( A \in \mathbb{R}^{m \times n} \), potem \( \frac{\partial A\tilde{x}}{\partial \tilde{x}} = A \).
    \item Če je \( \tilde{a} \in \mathbb{R}^n \), potem \( \frac{\partial \tilde{a}^T\tilde{x}}{\partial \tilde{x}} = \tilde{a}^T \).
    \item Če je \( A \in \mathbb{R}^{n \times n} \), potem \( \frac{\partial (\tilde{x}^T A\tilde{x})}{\partial \tilde{x}} = \tilde{x}^T(A + A^T) \).
    \item Če je \( A \in \mathbb{R}^{n \times n} \) simetrična matrika, potem velja \( \frac{\partial (\tilde{x}^T A\tilde{x})}{\partial \tilde{x}} = 2\tilde{x}^T A \).
    \item \( \frac{\partial \|\tilde{x}\|^2}{\partial \tilde{x}} = 2\tilde{x}^T \).
    \item Če \( \tilde{z} = \tilde{z}(\tilde{x}) \) in \( \tilde{y} = \tilde{y}(\tilde{x}) \), potem \( \frac{\partial (\tilde{y}^T \tilde{z})}{\partial \tilde{x}} = \tilde{y}^T \frac{\partial \tilde{z}}{\partial \tilde{x}} + \tilde{z}^T \frac{\partial \tilde{y}}{\partial \tilde{x}} \).
    \item Če \( G: D_G \subseteq \mathbb{R}^m \rightarrow \mathbb{R}^n \) in \( F: D_F \subseteq \mathbb{R}^n \rightarrow \mathbb{R}^p \) in \( H = F \circ G \), potem \( \frac{\partial H}{\partial \tilde{x}} = \frac{\partial F}{\partial G} (\tilde{G}(\tilde{x})) \frac{\partial G}{\partial \tilde{x}} \).
\end{enumerate}


\subsubsection{Izrek} 
Dvakrat zvezno odvedljiva funkcija \( f: D \subseteq \mathbb{R}^n \rightarrow \mathbb{R} \) je konveksna natanko tedaj, 
ko je \( \frac{\partial^2 f}{\partial \tilde{x}^2} \) pozitivno semidefinitna matrika na \( D \), in je konkavna natanko tedaj, 
ko je \( \frac{\partial^2 f}{\partial \tilde{x}^2} \) negativno semidefinitna matrika na \( D \).

\subsubsection{Prirejene funckije}

Naj bodo \( f, g_i, h_j: \mathbb{R}^n \rightarrow \mathbb{R} \) dane funkcije več spremenljivk. Radi bi našli rešitev naslednjega problema
\[
\text{(P)}^* \quad \min f(\vec{x})
\]
pri pogojih
\[
g_i(\vec{x}) \leq 0 \quad za \ i = 1,2,\ldots,m
\]
\[
h_j(\vec{x}) = 0 \quad za \ j = 1,2,\ldots,r.
\]

Definirajmo še množice \( D_g \), \( D_h \):
\[
D_g = \left\{ \vec{x} \in \mathbb{R}^n : g_i(\vec{x}) \leq 0 \quad za \ i = 1,2,\ldots,m \right\},
\]

\[
D_h = \left\{ \vec{x} \in \mathbb{R}^n : h_j(\vec{x}) = 0 \quad za \ j = 1,2,\ldots,r \right\}
\]

\[
D = D_f \cap \left( \bigcap_{i=1}^m D_g \right) \cap \left( \bigcap_{j=1}^r D_h \right).
\]

Sedaj lahko problem \( (P^*) \) zapišemo ekvivalentno kot
\[
    \text{(P)}^* \quad \min_{\vec{x} \in D} f(\vec{x}).
\]

Definirajmo Lagrangevo funkcijo
\[
L(\vec{x}, \vec{\lambda}, \vec{\mu}) = f(\vec{x}) - \vec{\lambda}^T \mathbf{G}(\vec{x}) - \vec{\mu}^T \mathbf{H}(\vec{x}) 
\]
\[
= f(\vec{x}) - \sum_{i=1}^{m} \lambda_i g_i(\vec{x}) - \sum_{j=1}^{r} \mu_j h_j(\vec{x}),
\]
kjer je
\[
\mathbf{G}(\vec{x}) = 
\begin{pmatrix}
g_1(\vec{x}) \\
\vdots \\
g_m(\vec{x})
\end{pmatrix},
\mathbf{H}(\vec{x}) = 
\begin{pmatrix}
h_1(\vec{x}) \\
\vdots \\
h_r(\vec{x})
\end{pmatrix},
\]
\[
\vec{\lambda} = 
\begin{pmatrix}
\lambda_1 \\
\vdots \\
\lambda_m
\end{pmatrix},
\vec{\mu} = 
\begin{pmatrix}
\mu_1 \\
\vdots \\
\mu_r
\end{pmatrix}.
\]

Funkcijo
\[
K(\vec{\lambda}, \vec{\mu}) = \inf_{\vec{x} \in D} L(\vec{x}, \vec{\lambda}, \vec{\mu}) = \inf_{\vec{x} \in D} \{ f(\vec{x}) - \vec{\lambda}^T \mathbf{G}(\vec{x}) - \vec{\mu}^T \mathbf{H}(\vec{x}) \}
\]
imenujemo \textbf{prirejena funkcija} problema  $(P^*) $. Pri tem spremenljivke $ \vec{\lambda} $ in $ \vec{\mu} $ imenujemo \textbf{prirejene spremenljivke}. Opazimo:

\begin{enumerate}
\item \( K(\vec{\lambda}, \vec{\mu}) \) je konkavna funkcija (neodvisno od lastnosti funkcij \( f, g_i, h_j \) originalnega problema).
\item Če je \( \lambda_i \leq 0 \) za \( i = 1,2,\ldots,m \), potem velja \( K(\vec{\lambda}, \vec{\mu}) \leq \min_{\vec{x} \in D} f(\vec{x}) \) za vse \( \vec{\lambda} \) ter vse \( \vec{\mu} \).
\end{enumerate}


\textbf{Problem}
\[
\text{(D}^*\text{)} \quad \max_{\vec{x},\vec{\lambda}, \vec{\mu}} K(\vec{\lambda}, \vec{\mu})
\]
pri pogojih
\[
\lambda_i \leq 0 \quad za \ i = 1,2,\ldots,m
\]
imenujemo \textbf{prirejeni problem} problema \( (P^*) \).

Označimo z \( \vec{x}^* \) vektor iz \( D \), ki reši problem \( (P^*) \) in \( \vec{\lambda}^* \), \( \vec{\mu}^* \) prirejene spremenljivke, ki rešita prirejeni problem \( (D^*) \). Naj bo torej \( p^* = f(\vec{x}^*) \) rešitev problema \( (P^*) \) in \( d^* = K(\vec{\lambda}^*, \vec{\mu}^*) \) rešitev problema \( (D^*) \). Potem iz (2) sledi
\[
d^* \leq p^*.
\]

Če
\begin{itemize}
\item je \( (P^*) \) linearni program (t.j., \( f \) je linearna in \( h_j \) so afine funkcije), ali če
\item so \( f, g_i \) konveksne funkcije in \( h_j \) afine,
\end{itemize}
potem velja \( d^* = p^* \).

V primeru, ko je \( d^* = p^* \), sledi, da morajo \( \vec{x}^*, \vec{\lambda}^* \) in \( \vec{\mu}^* \) zadostiti Karush-Kuhn-Tuckerjevim pogojem:
\[
\begin{aligned}
&\frac{\partial L(\vec{x}^*, \vec{\lambda}^*, \vec{\mu}^*)}{\partial \vec{x}} = 0, \\
&g_i(\vec{x}^*) \leq 0 \quad za \ i = 1,2,\ldots,m, \\
&h_j(\vec{x}^*) = 0 \quad za \ j = 1,2,\ldots,r, \\
&\lambda_i^* \leq 0 \quad za \ i = 1,2,\ldots,m, \\
&\lambda_i^* g_i(\vec{x}^*) = 0 \quad za \ i = 1,2,\ldots,m.
\end{aligned}
\]
(KKT)


\subsection{Dodatek 1: Vrste}

\textbf{Geometrijska vrsta}
\[
	\lim_{n \to \infty} S_n = \lim_{n \to \infty} \frac{a_1(1-q^n)}{1-q}, \quad |q|<1
\]
\[
	\lim_{n \to \infty} S_n = \frac{a_1}{1-q}, \quad |q|<1
\]

\textbf{Pravila za računanje z vrstami}
\[
	\sum a_n, \sum b_n \text{ konvergentni, potem velja:}
\]
\[
	\sum (a_n \pm b_n) = \sum a_n \pm \sum b_n
\]
\[
	\alpha \in \mathbb{R}, \sum (\alpha a_n) = \alpha \sum a_n
\]

\textbf{Dominirana konvergenca / divergenca}
\[
	\sum a_n, \sum b_n \text{ vrsti z nenegativnimi členi}
\]
\[
	a_n \geq 0, b_n \geq 0, a_n \leq b_n \ \forall n \in \mathbb{N}
\]
\[
	\text{če velja } \sum b_n \text{ konvergira, potem tudi } \sum a_n
\]
\[
	\text{če je } \sum b_n \text{ divergentna, je divergentna tudi } \sum a_n
\]

\textbf{Kvocientni kriterij}
\[
	\sum a_n, a_n>0 \ \forall n \in \mathbb{N}, D_n = \frac{a_{n+1}}{a_n}
\]
\[
	D = \lim_{n \to \infty} D_n = \lim_{n \to \infty} \frac{a_{n+1}}{a_n}
\]
\[
	D \leq 1 \Rightarrow \sum a_n \text{ konvergira}
\]
\[
	D > 1 \Rightarrow \sum a_n \text{ divergira}
\]
\[
	D = 0 \Rightarrow \text{ne moremo sklepati}
\]

\textbf{Korenski kriterij}
\[
	\sum a_n, a_n>0 \ \forall n \in \mathbb{N}, C_n = \sqrt[n]{a_n}
\]
\[
	C = \lim_{n \to \infty} C_n = \lim_{n \to \infty} \sqrt[n]{a_n}
\]
\[
	C < 1 \Rightarrow \sum a_n \text{ konvergira}
\]
\[
	C > 1 \Rightarrow \sum a_n \text{ divergira}
\]
\[
	C = 1 \Rightarrow \text{ne moremo sklepati}
\]

\textbf{Absolutna konvergenca}
\[
	\sum |a_n| \text{ konvergira } \Rightarrow \sum a_n \text{ konvergira}
\]
\[
	\sum a_n \text{ konvergira absolutno, če } \sum |a_n| \text{ konvergira}
\]

\textbf{Hipergeometrična vrsta}
\[
	\sum \frac{1}{n^p} \text{ konvergira, } \Leftrightarrow p>1 \ (p \in \mathbb{R})
\]

\textbf{Primerjalni kriterij 2}
\[
	\sum b_n \text{ abs. konvergentna in } |a_n| \leq |b_n|
\]
\[
	\Rightarrow \sum a_n \text{ abs. konvergentna}
\]

\textbf{Raabejev kriterij}
\[
	\sum a_n, a_n>0, R_n = n \left( \frac{a_n}{a_{n+1}} - 1 \right)
\]
\[
	R = \lim_{n \to \infty} R_n = \lim_{n \to \infty} n \left( \frac{a_n}{a_{n+1}} - 1 \right)
\]
\[
	R < 1 \Rightarrow \sum a_n \text{ divergira}
\]
\[
	R > 1 \Rightarrow \sum a_n \text{ konvergira}
\]
\[
	R = 1 \Rightarrow \text{ne moremo sklepati}
\]

\textbf{Leibnizov kriterij}
\[
	\sum (-1)^n a_n, a_n>0 \ \forall n \in \mathbb{N}, \text{ alternirajoča vrsta}
\]
\[
	\lim_{n \to \infty} a_n = 0 \text{ in } a_{n+1} \leq a_n \ \forall n \Rightarrow \sum (-1)^n a_n \text{ konvergira}
\]

\textbf{Primerjalni kriterij}
\[
	\sum b_n \text{ absolutno konvergentna in } |a_n| \leq |b_n| \text{ za vse } n \geq n_0.
\]
\[
	\Rightarrow \sum a_n \text{ je absolutno konvergentna}
\]


\subsection{Dodatek 2: Ponovitev analize}


\textbf{Odvodi}
\begin{center}
    \begin{small}
        \begin{enumerate}
            \item \begin{math}
                \frac{1}{x} = -\frac{1}{x^2}
            \end{math}
            \item \begin{math}
                x^n  = nx^{n-1}
            \end{math}
            \item \begin{math}
                \sqrt{x} = \frac{1}{2 \sqrt{x}}
            \end{math}
            \item \begin{math}
                \sqrt[n] x = \frac{1}{n \sqrt[n]{x^{n-1}}}
            \end{math}
            \item \begin{math}
                \sin (a x) =  a  \cos a x
            \end{math}
            \item  \begin{math}
                \cos (a x) = - a \sin (a x)
            \end{math}
            \item \begin{math}
                \tan x = \frac{1}{\cos^2 x} 
            \end{math}
            \item \begin{math}
                e^ax = ae^{ax}
            \end{math}
            \item \begin{math}
                a^x = a^x \ln a
            \end{math}
            \item \begin{math}
                x^x = x^x (1+\ln x)
            \end{math}
            \item \begin{math}
                ln x = \frac{1}{x}
            \end{math}
            \item \begin{math}
                log_a x = \frac{1}{x \ln a}
            \end{math}
            \item \begin{math}
                \arcsin x = \frac{1}{\sqrt {1 - x^2}}
            \end{math}
            \item \begin{math}
                \arccos x = - \frac{1}{\sqrt{1 - x^2}}
            \end{math}
            \item \begin{math}
                \arctan x = \frac{1}{1 + x^2}
            \end{math}
            \item \begin{math}
                \operatorname{arccot}x = -\frac{1}{1 + x^2}
            \end{math}
        \end{enumerate}
    \end{small}
\end{center}
\textbf{Integrali}
\begin{center}
    \begin{small}
        \begin{enumerate}
            \item \begin{math}
                \int x^a\,dx =
                \Bigg\{\begin{tabular}{ccc}
                    $\frac{x^{a+1}}{a+1} + C$  & $a \neq -1$ & \\
                    $ \ln{\left|x\right|} + C$ & $a = -1$ & \\
                  \end{tabular}
            \end{math}
            \item \begin{math}
                \int \ln {x}\,dx = x \ln {x} - x + C
            \end{math}
            \item \begin{math}
                \int \frac {1}{\sqrt{x}}\,dx=2\sqrt{x} + C 
            \end{math}
            \item \begin{math}
                \int e^x\,dx = e^x + C
            \end{math}
            \item \begin{math}
                \int a^x\,dx = \frac{a^x}{\ln{a}} + C
            \end{math}
            \item \begin{math}
                \int \cos({ax}) \, dx = { \frac{sin (ax)}{a} } + C
            \end{math}
            \item \begin{math}
                \int \sin({ax}) \, dx = { \frac{-cos(ax)}{a} } + C
            \end{math}
            \item \begin{math}
                \int \tan{x} \, dx = -\ln{\left| \cos {x} \right|} + C
            \end{math}
            \item \begin{math}
                \int \frac{dx}{\cos^2 x}=\int \sec^2 x \, dx = \tan x + C
            \end{math}
            \item \begin{math}
                \int \frac{dx}{\sin^2 x}=\int \csc^2 x \, dx = -\cot x + C
            \end{math}
            \item \begin{math}
                \int {\frac{1}{\sqrt{1-x^2}}} \, dx = \arcsin {x} + C
            \end{math}
            \item \begin{math}
                \int \frac{dx}{ax + b} = \frac{1}{a} ln |ax + b| + C
            \end{math}
            \item \begin{math}
                \int \frac{1}{x^2 + 1} \, dx = arctan x + C
            \end{math}
            \item \begin{math}
                \int \frac{dx}{x^2 + a^2} = \frac{1}{a} arctan \frac{x}{a} + C
            \end{math}
            \item \begin{math}
                \int \frac{f'(x)}{f(x)} \, dx = ln|f(x)| + C
            \end{math}
        \end{enumerate}
    \end{small}
\end{center}
    \textbf{Integriranje absolutnih vrednosti} (primer):
    \begin{small}
        Imamo funkcijo $f(x) = |x|$ , ki je zvezna na intervalu $[-1, 1]$
        Ce hocemo to funkcijo integrirati in zelimo izracunati njeno 
        \textit{porazdelitveno} funkcijo integrirati locimo 2 primera:    
    \end{small}
    \begin{enumerate}
        \item \begin{math}
            -1 \leq x < 0\\
            F(x) = \int_{-1}^x |t|\,dt = \int_{-1}^x -t\,dt = - \frac{t^2}{2} \rvert_{-1}^{x} = - \frac{1}{2} (x^2 -1)
        \end{math}
        \item \begin{math}
            0 \leq x < 1 \\
            F(x) = \int_{-1}^x |t|\,dt = \int_{-1}^0 -t\,dt + \int_{0}^x t\,dt = - \frac{t^2}{2} \rvert_{0}^{-1} + - \frac{t^2}{2} \rvert_{0}^{x}= \frac{1}{2} (1 + x^2)
        \end{math}
    \end{enumerate}
% other useful formulas
\begin{small}
    \begin{math}
        \sqrt[n] x^m = (x)^{\frac{m}{n}}, x^2 + y^2 \leq 1 \sim krog\; s\; ploscino\; \pi
    \end{math}
\end{small}


\end{multicols}
\end{document}

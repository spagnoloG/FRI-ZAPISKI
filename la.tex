\documentclass{article}
\usepackage[margin=0.2cm]{geometry}
\usepackage{amsmath}
\usepackage{multicol}
\usepackage{lipsum}% dummy text

\setlength{\columnseprule}{0.4pt}

\begin{document}
\begin{multicols}{3}

\section{\underline{Vektorji in matrike}}

\textbf{1.1} Vektor je \textit{urejena n-terica stevil}, ki jo obicajno
zapisemo kot stolpec\smallskip
\begin{center}
    $\vec{x}$ =
    $\begin{bmatrix}
        x_{1}\\
        \vdots \\
        x_{n}\\
    \end{bmatrix}$
\end{center}

\textbf{1.2} Produkt \textit{vektorja} $\vec{x}$ s skalarjem $\alpha$ je vektor
\begin{center}
    $\alpha \vec{x}$ =
    $\alpha$
    $\begin{bmatrix}
        x_{1}\\
        \vdots \\
        x_{n}\\
    \end{bmatrix}$ =
    $\begin{bmatrix}
        \alpha x_{1}\\
        \vdots \\
        \alpha x_{n}\\
    \end{bmatrix}$
\end{center}

\textbf{1.3} Vsota \textit{vektorjev} $\vec{x}$ in $\vec{y}$ je vektor
\begin{center}
    $\vec{x} + \vec{y} = 
    \begin{bmatrix}
        x_{1}\\
        \vdots \\
        x_{n}\\
    \end{bmatrix} +
    \begin{bmatrix}
        y_{1}\\
        \vdots \\
        y_{n}\\
    \end{bmatrix} =
    \begin{bmatrix}
        x_{1}  +  y_{1}\\
        \vdots\\
        x_{n} + y_{n}\\
    \end{bmatrix} 
    $
\end{center}

\textbf{1.4} Nicelni vektor $\vec{0}$ je tisti vektor, za katerega
je $\vec{a} + \vec{0} = \vec{0} + \vec{a} = \vec{a}$ za vsak vektor
$\vec{a}$. Vse komponente nicelnega vektorja so enake 0. Vsakemu vektorju
$\vec{a}$ priprada nasprotni vektor -$\vec{a}$, tako da je $\vec{a} + (-\vec{a}) = \vec{0}$
Razlika vektorjev $\vec{a}$ in $\vec{b}$ je vsota $\vec{a} + (-\vec{b})$ in jo
navadno zapisemo kot  $\vec{a} - \vec{b}$.

\textbf{Lastnosti vektorske vsote}
\begin{itemize}
    \item $\vec{a} + \vec{b} = \vec{b} + \vec{a}$ (komutativnost)
    \item $\vec{a} + (\vec{b} + \vec{c}) = (\vec{a} + \vec{b}) + \vec{c}$ (asociativnost)
    \item $a(\vec{a} + \vec{b}) = a\vec{a} + a\vec{b}$ (distributivnost)
\end{itemize}

\textbf{1.5} Linearna kombinacija vektorjev $\vec{x}$ in $\vec{y}$ je vsota
\begin{center}
    $a\vec{x} + b\vec{y}$
\end{center}

\textbf{1.6} Skalarni produkt vektorjev $\begin{bmatrix} x_{1}\\ \vdots \\ x_{n}\\ \end{bmatrix}$
in $\begin{bmatrix} y_{1}\\ \vdots \\ y_{n}\\ \end{bmatrix}$ je stevilo
\begin{center}
    $\vec{x} \cdot \vec{y} = x_{1}y_{1} + x_{2}y_{2} + \dots + x_{n}y_{n}$
\end{center}

\textbf{Lastnosti skalarnega produkta}
\begin{itemize}
    \item $\vec{x} \cdot \vec{y} = \vec{y} \cdot \vec{x}$ (komutativnost)
    \item $\vec{x} \cdot (\vec{y} + \vec{z}) = \vec{x} \cdot \vec{y} + \vec{x} \cdot \vec{z}$ (aditivnost)
    \item $\vec{x} \cdot (a \vec{y}) = a(\vec{x} \cdot \vec{y}) = (a \vec{x}) \cdot \vec{y}$ (homogenost)
    \item $\forall \vec{x}$ \textit{velja} $\vec{x} \cdot \vec{x} \geq 0$
\end{itemize}

\textbf{1.7} Dolzina vektorja $\vec{x}$ je
\begin{center}
    $||\vec{x}|| = \sqrt{\vec{x} \cdot \vec{x}}$
\end{center}

\textbf{1.8} Enotski vektor je vektor z dolzino 1.

\textbf{1.9} Za poljubna vektorja $\vec{u}, \vec{v} \in R^{n}$ velja:
\begin{center}
    $|\vec{u} \cdot \vec{v}| \leq ||\vec{u}||||\vec{v}||$.
\end{center}

\textbf{1.10} Za poljubna vektorja $\vec{u}, \vec{v} \in R^{n}$ velja:
\begin{center}
    $||\vec{u} + \vec{v}|| \leq ||\vec{u}||+||\vec{v}||$.
\end{center}

\textbf{1.11} Vektorja $\vec{x}$ in $\vec{y}$ sta ortogonalna
(ali pravokotna) natakno takrat, kadar je
\begin{center}
    $\vec{x} \cdot \vec{y} = $ 0    
\end{center}

\textbf{1.12} Ce je $\phi$ kot med vektorjema $\vec{x}$ in $\vec{y}$, potem je
\begin{center}
    $\dfrac{\vec{x} \cdot \vec{y}}{||\vec{x}|| ||\vec{y}||} =
    \cos \phi$
\end{center}

\textbf{1.13} Vektorski produkt:
\begin{center}
    $\vec{a} \times \vec{b} = (a_{2}b_{3} - a_{3}b_{2}) \textbf{i}$ +
    $(a_{3}b_{1} - a_{1}b_{3}) \textbf{j} + (a_{1}b_{2} - a_{2}b_{1}) \textbf{k}$
\end{center}

\textbf{Lastnosti vektorskega produkta}
\begin{itemize}
    \item $\vec{a} \times (\vec{b} + \vec{c}) = \vec{a} \times \vec{b} + \vec{a} \times \vec{c}$ (aditivnost)
    \item $\vec{b} \times \vec{a} = -\vec{a} \times \vec{b}$ (!komutativnost)
    \item $ (a \vec{a}) \times \vec{b} = a(\vec{a} \times \vec{b}) =  \vec{a} \times (a \vec{b})$ (homogenost)
    \item $\vec{a} \times \vec{a} = 0$
    \item $\vec{a} \times \vec{b}$  \textit{je}  $\perp$ \textit{na vektorja} $\vec{a}$ \textit{in} $\vec{b}$
    \item $||\vec{a} \times \vec{b}|| = ||\vec{a}|| ||\vec{b}|| \sin \phi$
    \item Dolzina vektorskega produkta je ploscina paralelograma, katerega vektorja oklepata 
\end{itemize}

\textbf{1.14} Mesani produkt($\vec{a}, \vec{b}, \vec{c}$) vektorjev
$\vec{a}, \vec{b}$ in $\vec{c}$ v $R^{3}$ je skalarni produkt vektorjev
$\vec{a} \times \vec{b}$ in $\vec{c}$:
\begin{center}
    $(\vec{a}, \vec{b}, \vec{c}) = (\vec{a} \times \vec{b})\cdot \vec{c}$
\end{center}

\textbf{Lastnosti mesanega produkta}
\begin{itemize}
    \item $(\vec{a}, \vec{b}, \vec{c}) = (\vec{b}, \vec{c}, \vec{a}) = (\vec{c}, \vec{a}, \vec{b})$
    \item $(x\vec{a}, \vec{b}, \vec{c}) = x(\vec{a}, \vec{b}, \vec{c})$ (homogenost)
    \item $(\vec{a}, \vec{u} + \vec{v}, \vec{c}) = (\vec{a}, \vec{u}, \vec{c}) + (\vec{a}, \vec{v}, \vec{c})$
    \item Absolutna vrednost mesanega produkta ($\vec{a}, \vec{b}, \vec{c}$) je enaka prostornini paralepipeda
\end{itemize}

\section{Test}
\lipsum
\section{Test}

\end{multicols}
\end{document}

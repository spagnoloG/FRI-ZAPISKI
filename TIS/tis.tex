\documentclass{article}
\usepackage[margin=0.15cm]{geometry}
\usepackage{amsmath}
\usepackage{multicol}

\setlength{\columnseprule}{0.5pt}

\begin{document}

\begin{center}
    {\small TIS/FRI \par}
\end{center}

\begin{multicols}{3}

\section{\underline{Osnove}}

\textbf{1.1 Ponovitev logaritmov}
\begin{itemize} 
    \item $log_a x = \dfrac{log_b x}{log_b a}$
    \item $log_b(\dfrac{x}{y}) = log_b x - log_b y$
    \item $x = b^y \implies log_b x = y$
    \item $log_2 x = log x$
    \item $0log0 = 0$
\end{itemize}

\textbf{1.2 Entropija}
je povprecje vseh lastnih informacij:
\begin{center}
    \begin{math}
        H(X) = \sum_{i=1}^{n} p_i I_i = -\sum_{i=1}^{n} p_i log p_i
    \end{math}
\end{center}
Lastnosti: je zvezna, simetricna funckija (vrsni red $p_i$ ni pomemben, sestevanje je komutativno). Je vedno vecja
od 0 ($p_i \geq 0 \rightarrow -p_i \log p_i \geq 0 \rightarrow H(X) \geq 0$) in navzgor omejena z $\log n$.\\
Ce sta dogodka \textbf{neodvisna} velja aditivnost: $H(X, Y) = H(X) + H(Y)$.\\
Vec zaporednih dogodkov neodvisnega vira: $X^l = X \times \dots \times X \rightarrow H(X^l) = lH(X)$.

\section{\underline{Kodi}}

\textbf{2.1 Uvod}\\
\textbf{Kod} sestavljajo \textit{kodne zamenjave}, ki so sestavljene iz znakov
\textbf{kodne abecede}. Stevilo znakov v kodni abecedi oznacujemo z \textbf{r}.\\
Ce so $\{p_1, \dots, p_n\}$ verjetnosti znakov $\{s_1, \dots, s_n\}$ osnovnega sporocila in $\{l_1, \dots, l_n\}$
dolzine prejetih kodnih zmanjav, je povprecna dolzina kodne zamenjave
\begin{center}
    \begin{math}
        L = \sum_{i=1}^n p_i l_i
    \end{math}
\end{center}

\textbf{2.2 Tipi kodov}
\begin{itemize}
    \item \textbf{optimalen} - ce ima najmanjso mozno dolzino kodnih zamenjav
    \item \textbf{idealen} - ce je povprecna dolzina kodnih zamenjav enaka entropiji
    \item \textbf{enakomeren} - ce je dolzina vseh kodnih zamenjav enaka
    \item \textbf{enoznacen} - ce lahko poljuben niz znakov dekodiramo na en sam nacin
    \item \textbf{trenuten} - ce lahko osnovni znak dekodiramo takoj, ko sprejmemo celotno kodno zamenjavo
\end{itemize}

\textbf{2.3 Kraftova neenakost}
Za dolzine kodnih zamenjav $\{l_1, \dots, l_n\}$ in \textit{r} znaki kodne abecede
obstaja trenutni kod, iff:
\begin{center}
    \begin{math}
        \sum_{i=1}^n r^{-li} \leq 1
    \end{math}
\end{center}
% 7 stran predavanje 2

\end{multicols}
\end{document}

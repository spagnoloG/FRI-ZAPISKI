\documentclass{article}
\usepackage[margin=0.15cm]{geometry}
\usepackage{amsmath}
\usepackage{multicol}

\setlength{\columnseprule}{0.5pt}

\begin{document}

\begin{center}
    {\small TIS/FRI \par}
\end{center}

\begin{multicols}{3}

\section{\underline{Osnove}}

\textbf{1.1 Ponovitev logaritmov}
\begin{small}
    \begin{itemize} 
        \item $log_a x = \dfrac{log_b x}{log_b a}$
        \item $log_b(\dfrac{x}{y}) = log_b x - log_b y$
        \item $x = b^y \implies log_b x = y$
        \item $log_2 x = log x$
        \item $0log0 = 0$
    \end{itemize}
\end{small}

\textbf{1.2 Entropija}
je povprecje vseh lastnih informacij:
\begin{center}
    \begin{math}
        H(X) = \sum_{i=1}^{n} p_i I_i = -\sum_{i=1}^{n} p_i log p_i
    \end{math}
\end{center}
Lastnosti: je zvezna, simetricna funckija (vrsni red $p_i$ ni pomemben, sestevanje je komutativno). Je vedno vecja
od 0 ($p_i \geq 0 \rightarrow -p_i \log p_i \geq 0 \rightarrow H(X) \geq 0$) in navzgor omejena z $\log n$.\\
Ce sta dogodka \textbf{neodvisna} velja aditivnost: $H(X, Y) = H(X) + H(Y)$.\\
Vec zaporednih dogodkov neodvisnega vira: $X^l = X \times \dots \times X \rightarrow H(X^l) = lH(X)$.

\section{\underline{Kodi}}

\textbf{2.1 Uvod}\\
\textbf{Kod} sestavljajo \textit{kodne zamenjave}, ki so sestavljene iz znakov
\textbf{kodne abecede}. Stevilo znakov v kodni abecedi oznacujemo z \textbf{r}.\\
Ce so $\{p_1, \dots, p_n\}$ verjetnosti znakov $\{s_1, \dots, s_n\}$ osnovnega sporocila in $\{l_1, \dots, l_n\}$
dolzine prejetih kodnih zmanjav, je povprecna dolzina kodne zamenjave
\begin{center}
    \begin{math}
        L = \sum_{i=1}^n p_i l_i
    \end{math}
\end{center}

\textbf{2.2 Tipi kodov}
\begin{itemize}
    \item \textbf{optimalen} - ce ima najmanjso mozno dolzino kodnih zamenjav
    \item \textbf{idealen} - ce je povprecna dolzina kodnih zamenjav enaka entropiji
    \item \textbf{enakomeren} - ce je dolzina vseh kodnih zamenjav enaka
    \item \textbf{enoznacen} - ce lahko poljuben niz znakov dekodiramo na en sam nacin
    \item \textbf{trenuten} - ce lahko osnovni znak dekodiramo takoj, ko sprejmemo celotno kodno zamenjavo
\end{itemize}

\textbf{2.3 Kraftova neenakost}
Za dolzine kodnih zamenjav $\{l_1, \dots, l_n\}$ in \textit{r} znaki kodne abecede
obstaja trenutni kod, iff
\begin{center}
    \begin{math}
        \sum_{i=1}^n r^{-li} \leq 1
    \end{math}
\end{center}

\textbf{2.4 Povprecna dolzina} in \textbf{ucinkovitost}\\
Najkrajse kodne zamenjave imamo, ce velja:
\begin{center}
    \begin{math}
        H_r(X) = L \rightarrow l_i = \lceil - \log_r p_i \rceil
    \end{math}
\end{center}
Ucinkovitost koda:
\begin{center}
    \begin{math}
        \eta = \dfrac{H(X)}{L \log_r}, \eta \in [0, 1]
    \end{math}
\end{center}
Kod je \textbf{gospodaren}, ce je $L$ znotraj:
\begin{center}
    \begin{math}
        H_r(X) \leq L < H_r(X) + 1
    \end{math}
\end{center}
kjer je $H_r(X)$:
\begin{center}
    \begin{math}
        H_r(X) = -\sum^{n}_{i=1} \dfrac{\log_{p_i}}{\log_r} = \dfrac{H(X)}{\log_r}
    \end{math}
\end{center}

\textbf{2.5 Shannonov prvi teorem}\\
Za nize neodvisnih znakov dozline \texit{n} obstajajo kodi,
za katere velja:
\begin{center}
    \begin{math}
        \lim_{n \rightarrow \infty} \dfrac{L_n}{n} = H(X)
    \end{math}
\end{center}
pri cemer je $H(X)$ entropija vira $X$.\\
Postopek kodiranja po Shannonu:
\begin{enumerate}
    \item znake razvrstimo po padajocih verjetnostih
    \item dolocimo stevilo znakov v vsaki kodni zamenjavi ($l_k$)
    \item za vse simbole izracunamo komulativne verjetnosti ($P_k = \sum_{i=1}^{k-1} p_i$)
    \item $P_k$ pretvorimo v bazo $r$. Kodno zamenjavo predstavlja prvih $l_k$ znakov necelega dela stevila
\end{enumerate}

\textbf{2.6 Fanojev kod}\\
Postopek kodiranja:
\begin{enumerate}
    \item znake razvrstimo po padajocih verjetnostih
    \item znake razdelimo v $r$ cim bolj enako verjetnih skupin
    \item Vsaki skupini priredimo enega od r znakov kodne abecede
    \item Deljenje ponovimo na vsaki od skupin. Postopek ponavljamo, dokler je mogoce
\end{enumerate}

\textbf{2.7 Huffmanov kod}\\
Huffmanov postopek kodiranja poteka od spodaj navzgor (Pri Fanoju je ravno obratno).
Pri huffmanovem kodu imamo dve fazi:
\begin{enumerate}
    \item Zdruzevanje
        \begin{enumerate}
            \item Posici r najmanj verjetnih znakov in jih zdruzi v sestavljeni znak, katerega verjetnost je vsota verjetnosti vseh znakov
            \item Preostale znake skupaj z novo sestavljenim znakom spet razvrsti
            \item Postopek ponavljaj dokler ne ostane samo r znakov
        \end{enumerate}
    \item Razdruzevanje
        \begin{enumerate}
            \item Vsakemu od preostalih znakov priredi po en znak kodirne abecede
            \item Vsak sestavljeni znak razstavi in mu priredi po en znak kodirne abecede
            \item Ko zmanjka sestavljenih znakov, je postopek zakljucen
        \end{enumerate}
\end{enumerate}
Pred kodiranjem, je vedno pametno preveriti, ce imamo zadostno stevilo znakov.
Veljati mora:
\begin{center}
    \begin{math}
        n = r + k(r-1), k \geq 0
    \end{math}
\end{center}
Ce imamo premalo znakov, jih po potrebi dodamo s verjetnostjo $p=0$.\\
Huffmanov kod lahko razsirimo tako, da vec osnovnih znakov zdruzujemo v sestavljene znake $\rightarrow$ bolj ucinkoviti kodi. Vendar
naletimo na nevarnost kombinacijske eksplozije.\\

\textbf{2.9 Aritmeticni kod}\\
Je \textbf{hiter} in \textbf{blizu optimalnemu} kodu. Je manj ucinkovit kot Huffmanov,
vendar se izogne kombinacijski eksploziji.
Vsak niz je predstavljen kot realno stevilo $0 \leq R < 1$, kar nam pove, da daljsi kot bo niz,
bolj natancno mora biti podano naravno stevilo $R$.\\
Postopek kodiranja(znakov ni potrebno razvrstiti):
\begin{enumerate}
    \item Zacnemo z intervalom $[0, 1]$
    \item Izbrani interval razdelimo na \texit{n} podintervalov, ki se ne prekrivajo. Sirine podintervalov ustrezajo
        verjetnostim znakov. Vsak podinterval predstavlja en znak
    \item Izberemo podintrval, ki ustreza iskanemu znaku
    \item Ce niz se ni koncan, izbrani podinterval ponovno razdelimo (bne 2.tocka)
    \item Niz lahko predstavimo s poljubnim realnim stevilom v zadnjem podintervalu
\end{enumerate}
Ko dobimo realni interal, ga samo se pretvorimo v binarnega s pomocjo klasicnega pretvarjanja
iz dec v bin stevilski sistem.
%Predavanje 04%
\end{multicols}
\end{document}
